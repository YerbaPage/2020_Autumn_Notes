\documentclass[10pt]{paper}
% \usepackage[utf8]{inputenc}
\usepackage{fancyhdr}
\pagestyle{plain}
\usepackage[english]{babel}
\usepackage{hyperref}
\usepackage{amstext}
\usepackage{amssymb}
\usepackage{amsmath}
\usepackage[a4paper,scale=0.9]{geometry}
\usepackage{xcolor}
\definecolor{winered}{rgb}{0.5,0,0}
\hypersetup{
  pdfborder={0 0 0},
  colorlinks=true,
  linkcolor={winered},
  urlcolor={winered},
  filecolor={winered},
  citecolor={winered},
  linktoc=all,
}

\newtheorem{theorem}{Theorem}[section]
\newtheorem{definition}{Def}[section]
\newtheorem{corollary}{Corollary}[section]
\newtheorem{lemma}[theorem]{Lemma}

\title{Linear Functional Analysis Notes}
\date{\today}
\author{{\footnotesize{Yuling Shi}} with Erin}
\begin{document}
\maketitle
\tableofcontents
\footnotesize
\section{Preliminaries}

\begin{itemize}
    \item Galerkin
    \item the norm exercise in class
    \item orthogonalization 
\end{itemize}

\subsection{Linear Algebra}

\begin{definition}[Linear Subspace]
    A non-empty set $U \subset V$ is a \textbf{linear subspace} of $V$ if $\alpha x+\beta y \in U $ for all $\alpha, \beta \in \mathbb{F}$ and $x, y \in U$
\end{definition}

\begin{definition}[Standard basis]
    $\widehat{e}_{1}=(1,0,0, \ldots, 0), \widehat{e}_{2}=(0,1,0, \ldots, 0), \ldots, \widehat{e}_{k}=(0,0,0, \ldots, 1)$ will be called the \textbf{standard basis} for $\mathbb{F}^{k}$
\end{definition}

\begin{definition}[Linear transformation]
    If for all $\alpha, \beta \in \mathbb{F}$ and $x, y \in V$: $T(\alpha x+\beta y)=\alpha T(x)+\beta T(y)$
\end{definition}

\subsection{Metric Spaces}

\begin{definition}[Metric]
    A metric on a set $M$ is a function for all $x, y, z \in M$
    \begin{itemize}
        \item $d(x, y) \geq 0$
        \item $d(x, y)=0 \Longleftrightarrow x=y$
        \item $d(x, y)=d(y, x)$
        \item $d(x, z) \leq d(x, y)+d(y, z) \quad$ (the triangle inequality)
    \end{itemize}
    If $d$ is a metric on $M,$ then the pair $(M, d)$ is called a \textbf{metric space}.
\end{definition}

\begin{definition}[Induce]
    Let $(M, d)$ be a metric space and let $N$ be a subset of $M .$ Define $d_{N}: N \times N \rightarrow \mathbb{R}$ by $d_{N}(x, y)=d(x, y)$ for all $x, y \in N$ (that is, $d_{N}$ is the restriction of $d$ to the subset $N$ ). Then $d_{N}$ is a metric on $N,$ called the metric induced on $N$ by $d$.
\end{definition}

\begin{theorem}
    Suppose that $\left\{x_{n}\right\}$ is a convergent in a metric space $(M, d)$ Then $\left\{x_{n}\right\}$ is also a Cauchy sequence.
\end{theorem}

\begin{definition}[Open ball]
    $B_{x}(r)=\{y \in M: d(x, y)<r\}$ will be called the open ball with centre $x$ and radius $r$
\end{definition}

\begin{definition}
    Let $(M, d)$ be a metric space and let $A \subset M$
    \begin{enumerate}
        \item $A$ is \textbf{bounded} if there is a number $b>0$ such that $d(x, y)<b$ for all $x, y \in A$
        \item $A$ is \textbf{open} if, for each point $x \in A,$ there is an $\epsilon>0$ such that $B_{x}(c) \subset A$
        \item $A$ is \textbf{closed} if the set $M \backslash A$ is open.
        \item  A point $x \in M$ is a \textbf{closure point} of $A$ if, for every $\epsilon>0,$ there is a point $y \in A$ with $d(x, y)<\epsilon$ (equivalently, if there exists a sequence $\left\{y_{n}\right\} \subset A$ such that $y_{n} \rightarrow x$ ).
        \item  The \textbf{closure} of $A,$ denoted by $\bar{A}$ or $A^{-},$ is the set of all closure points of $A$.
        \item  $A$ is \textbf{dense} $($ in $M)$ if $\bar{A}=M$.
    \end{enumerate}
\end{definition}

\begin{theorem}
    Let $(M, d)$ be a metric space and let $A \subset M$.
    \begin{enumerate}
        \item $\bar{A}$ is closed and is equal to the intersection of the collection of all closed subsets of $M$ which contain $A$ (so $\bar{A}$ is the smallest closed set containing
              $A)$
        \item $A$ is closed if and only if $A=\bar{A}$.
        \item $A$ is closed if and only if, whenever $\left\{x_{n}\right\}$ is a sequence in $A$ which converges to an element $x \in M,$ then $x \in A$
        \item $x \in \bar{A}$ if and only if $\inf \{d(x, y): y \in A\}=0$
        \item $A$ is dense if and only if, for any element $x \in M$ and any number $\epsilon>0$, there exists a point $y \in A$ with $d(x, y)<\epsilon$ (equivalently, for any element $x \in M$ there exists a sequence $\left\{y_{n}\right\} \subset A$ such that $\left.y_{n} \rightarrow x\right)$
    \end{enumerate}
\end{theorem}

\begin{definition}
    $f$ is \textbf{uniformly continuous} (on $M$ ) if, for every $\epsilon>0,$ there exists $\delta>0$ such that, for all $x, y \in M$ $d_{M}(x, y)<\delta \Rightarrow d_{N}(f(x), f(y))<\epsilon$.
\end{definition}



\begin{theorem}
    Suppose that $(M, d)$ is a metric space and $A \subset M .$ Then:
    \begin{enumerate}
        \item if $A$ is complete then it is closed;
        \item if $M$ is complete then $A$ is complete if and only if it is closed;
        \item if $A$ is compact then it is closed and bounded;
        \item (Bolzano-Weierstrass theorem) every closed, bounded subset of $\mathbb{F}^{k}$ is compact.
    \end{enumerate}
\end{theorem}

\begin{theorem}
    $f$ is continuous at $x \in M$ if and only if $x_{n} \rightarrow x,$ satisfies $f\left(x_{n}\right) \rightarrow f(x)$, $f$ is continuous on $M$ if and only if either any open set $A \subset N,$ the set $f^{-1}(A) \subset M$ is open or for any closed set $A \subset N,$ the set $f^{-1}(A) \subset M$ is closed.
\end{theorem}

\begin{definition}[complete]
    A metric space $(M, d)$ is \textbf{complete} if every Cauchy sequence in it is convergent. A set $A \subset M$ is \textbf{complete} (in $(M, d))$ if every Cauchy sequence in $A$ converges to an element of $A$. The space $\mathbb{F}^{k}$ with the standard metric is complete.
\end{definition}

\begin{theorem}[Baire's Category Theorem]
    If $(M, d)$ is a complete metric space and $M=\bigcup_{j=1}^{\infty} A_{j},$ where each $A_{j} \subset M$ $j=1,2, \ldots,$ is closed, then at least one of the sets $A_{j}$ contains an open ball.
\end{theorem}

\begin{definition}[Compact]
    Let $(M, d)$ be a metric space. A set $A \subset M$ is \textbf{compact} if every sequence $\left\{x_{n}\right\}$ in $A$ contains a subsequence that converges to an element of $A .$ A set $A \subset M$ is \textbf{relatively compact} if the closure $\bar{A}$ is compact. If the set $M$ itself is compact then we say that $(M, d)$ is a \textbf{compact metric space}.
\end{definition}

\begin{theorem}
    Suppose that $(M, d)$ is a metric space and $A \subset M .$ Then:
    \begin{enumerate}
        \item if $A$ is complete then it is closed;
        \item if $M$ is complete then $A$ is complete if and only if it is closed;
        \item if $A$ is compact then it is closed and bounded;
        \item (Bolzano-Weierstrass) every closed, bounded subset of $\mathbb{F}^{k}$ is compact.
    \end{enumerate}
\end{theorem}

\begin{theorem}
    Suppose that $(M, d)$ is a compact metric space and $f: M \rightarrow \mathbb{F}$ is continuous. Then there exists a constant $b>0$ such that $|f(x)| \leq b$ for all $x \in M$ (we say that $f$ is \textbf{bounded}). In particular, if $\mathbb{F}=\mathbb{R}$ then the numbers $\sup \{f(x): x \in M\}$ and $\inf \{f(x): x \in M\},$ exist and are finite. Furthermore, there exist points $x_{s}, x_{i} \in M$ such that $f\left(x_{s}\right)=\sup \{f(x): x \in M\}, f\left(x_{i}\right)=\inf \{f(x): x \in M\}$
\end{theorem}

\begin{definition}
    Let $(M, d)$ be a compact metric space. Set of continuous functions $f: M \rightarrow$ $\mathbb{F}$ will be denoted by $C_{\mathbb{F}}(M) .$ We define \textbf{uniform metric} on $C_{\mathrm{F}}(M)$ by
    $$
        d(f, g)=\sup \{|f(x)-g(x)|: x \in M\}
    $$
\end{definition}

\begin{definition}
    $(M, d)$ is a compact metric space and $\left\{f_{n}\right\}$ is a sequence in $C(M)$
    \begin{enumerate}
        \item $\left\{f_{n}\right\}$ converges pointwise to $f$ if $\left|f_{n}(x)-f(x)\right| \rightarrow 0$ for all $x \in M$
        \item $\left\{f_{n}\right\}$ converges uniformly to $f$ if $\sup \left\{\left|f_{n}(x)-f(x)\right|: x \in M\right\} \rightarrow 0$.
    \end{enumerate}
\end{definition}

\begin{theorem}
    The metric space $C_{\mathbb{F}}(M)$ (also written as $C(M)$) is complete. For any compact set $M \subset \mathbb{R},$ the set $\mathcal{P}_{\mathrm{R}}$ (polynomial) is dense in $C_{\mathbb{R}}(M)$.
\end{theorem}

\begin{definition}[Separable]
    A metric space $(M, d)$ is \textbf{separable} if it contains a countable, dense subset. The empty set is regarded as separable.
\end{definition}

\begin{theorem}
    Suppose that $(M, d)$ is a metric space and $A \subset M$, then:
    \begin{enumerate}
        \item If $A$ is compact then it is separable.
        \item If $A$ is separable and $B \subset A$ then $B$ is separable.
    \end{enumerate}
\end{theorem}

\subsection{Lebesgue Integration}

\begin{definition}[counting measure]
    We define counting measure
    $$\mu(A)=\left\{\begin{array}{ll}
            n      & \text{if A has exactly } n \text { elements } \\
            \infty & \text { otherwise }
        \end{array}\right.$$
\end{definition}

\begin{definition}[Essential supremum]
    Suppose that $f$ is a measurable function and there exists a number $b$ such that $f(x) \leq b a.e.$ Then we can define the \textbf{essential supremum} of $f$ to be $$ess \ sup \  f=\inf \{b: f(x) \leq b \  a.e. \}$$
\end{definition}

\begin{definition}
    We define an equivalence relation $\equiv$ on $\mathcal{L}^{1}(X)$ by $$f \equiv g \Longleftrightarrow f(x)=g(x) \text { for a.e. } x \in X$$ This relation partitions the set $\mathcal{L}^{1}(X)$ into a space of equivalence classes, which we will denote by $L^{1}(X)$.
\end{definition}

\begin{definition}
    Suppose that $1 \leq p \leq \infty .$ In the special case where $(X, \Sigma, \mu)=\left(\mathbb{N}, \Sigma_{c}, \mu_{c}\right),$ the space $L^{p}(\mathbb{N})$ consists of the set of sequences $\left\{a_{n}\right\}$ in $\mathbb{F}$ with the property that
    $$
        \begin{array}{l}
            \left(\sum_{n=1}^{\infty}\left|a_{n}\right|^{p}\right)^{1 / p}<\infty, \quad \text { for } 1 \leq p<\infty \\
            \sup \left\{\left|a_{n}\right|: n \in \mathbb{N}\right\}<\infty, \quad \text { for } p=\infty
        \end{array}
    $$
    These spaces will be denoted by $\ell^{p}$ (or $\ell_{k}^{p}$ )
\end{definition}

\begin{definition}
    Define the spaces
    $$
        \begin{array}{l}
            \mathcal{L}^{p}(X)=\left\{f: f \text { is measurable and }\left(\int_{X}|f|^{p} d \mu\right)^{1 / p}<\infty\right\}, \quad 1 \leq p<\infty \\
            \mathcal{L}^{\infty}(X)=\{f: f \text { is measurable and ess sup }|f|<\infty\}
        \end{array}
    $$
    We also define the corresponding sets $L^{P}(X)$ by identifying functions in $\mathcal{L}^{p}(X)$ which are $a.e.$ equal. In practice, we simply refer to such $\mathcal{L}$ rather than classes themselves.
\end{definition}

\begin{theorem}[Two useful inequalities]
    Minkowski's inequality $($ for $1 \leq p<\infty)$
    $$
        \begin{array}{c}
            \left(\int_{X}|f+g|^{p} d \mu\right)^{1 / p} \leq\left(\int_{X}|f|^{p} d \mu\right)^{1 / p}+\left(\int_{X}|g|^{p} d \mu\right)^{1 / p} \\
            ess\  \sup |f+g| \leq ess\  \sup |f|+ess\  \sup |g|
        \end{array}
    $$
    Hölder's inequality (for $1<p<\infty$ and $p^{-1}+q^{-1}=1$ ):
    $$
        \begin{array}{l}
            \int_{X}|f g| d \mu \leq\left(\int_{X}|f|^{p} d \mu\right)^{1 / p}\left(\int_{X}|g|^{q} d \mu\right)^{1 / q} \\
            \int_{X}|f g| d \mu \leq ess\  \sup |f| \int_{X}|g| d \mu
        \end{array}
    $$
\end{theorem}


\begin{theorem}
    Suppose that $1 \leq p \leq \infty .$ Then the metric space $L^{P}(X)$ is complete. In particular, the sequence space $\ell^{P}$ is complete.
\end{theorem}

\begin{theorem}
    Suppose that $[a, b]$ is a bounded interval and $1 \leq p<\infty .$ Then the set $C[a, b]$ is dense in $L^{p}[a, b]$
\end{theorem}



\section{Normed Spaces}
\subsection{Examples of Normed Spaces}
\begin{definition}[Norm]
    Let $X$ be a vector space over $\mathbb{F}$. A norm on $X$ is a function $\|\cdot\|: X \rightarrow \mathbb{R}$ such that for all $x, y, \in X$ and $\alpha \in \mathbb{F}$
    \begin{enumerate}
        \item $\|x\| \geq 0$
        \item $\|x\|=0$ if and only if $x=0$
        \item $\|\alpha x\|=\|\alpha\|\|x\|$
        \item $\|x+y\| \leq\|x\|+\|y\|$
    \end{enumerate}
    A vector space X on which there is a norm is called a normed vector space or just a normed space. If $X$ is a normed space, a unit vector in $X$ is a vector $x$ such that $\|x\|=1$
\end{definition}

\begin{lemma}
    Let $X$ be a vector space with norm $\|\cdot\| .$ If $d: X \times X \rightarrow \mathbb{R}$ is defined by $d(x, y)=\|x-y\|$ then $(X, d)$ is a metric space.
\end{lemma}

\begin{definition}[Metric]
    If $X$ is a vector space with norm $\|\cdot\|$ and $d$ is the metric defined by $d(x, y)=$ $\|x-y\|$ then $d$ is called the metric associated with $\|\cdot\| \cdot$.
\end{definition}

\begin{definition}[Convergence]
    We are trying to define convergence in our normed space. We will have strong convergence as $\lim _{n \rightarrow \infty}\left\|x_{n} - x\right\|=0$ which is also written as $\lim _{n \rightarrow \infty}x_{n}=x$. And we also have weak convergence as $\lim _{n \rightarrow \infty}\left\|x_{n}\right\|=\|x\|$.
\end{definition}

\begin{theorem}
    Let $X$ be a vector space over $\mathbb{F}$ with norm $\|\cdot\| .$ Let $\left\{x_{n}\right\}$ and $\left\{y_{n}\right\}$ be sequences in $X$ which converge to $x, y$ in $X$ respectively and let $\left\{\alpha_{n}\right\}$ be a sequence in $\mathbb{F}$ which converges to $\alpha$ in $\mathbb{F}$. Then:
    \begin{enumerate}
        \item $|\|x\|-\|y\|| \leq\|x-y\|$
        \item $\lim _{n \rightarrow \infty}\left\|x_{n}\right\|=\|x\|$
        \item $\lim _{n \rightarrow \infty}\left(x_{n}+y_{n}\right)=x+y$
        \item $\lim _{n \rightarrow \infty} \alpha_{n} x_{n}=\alpha x$
    \end{enumerate}

\end{theorem}

\subsection{Finite-dimensional Normed Spaces}
\begin{definition}[Equivalence]
    Let $X$ be a vector space and let $\|\cdot\|_{1}$ and $\|\cdot\|_{2}$ be two norms on $X .$ The norm $\|\cdot\|_{2}$ is equivalent to the norm $\|\cdot\|_{1}$ if there exists $M, m>0$ such that for all $x \in X$
    $$
        m\|x\|_{1} \leq\|x\|_{2} \leq M\|x\|_{1}
    $$
\end{definition}

\begin{lemma}
    Let $X$ be a vector space and let $\|\cdot\|_{1},\|\cdot\|_{2}$ and $\|\cdot\|_{3}$ be three norms on $X$. Let $\|\cdot\|_{2}$ be equivalent to $\|\cdot\|_{1}$ and let $\|\cdot\|_{3}$ be equivalent to $\|\cdot\|_{2}$
    \begin{enumerate}
        \item $\|\cdot\|_{1}$ is equivalent to $\|\cdot\|_{2}$ (exchanging position)
        \item $\|\cdot\|_{3}$ is equivalent to $\|\cdot\|_{1}$ (passing)
    \end{enumerate}
\end{lemma}

\begin{lemma}
    Let $X$ be a vector space and let $\|\cdot\|$ and $\|\cdot\|_{1}$ be norms on $X .$ Let $d$ and $d_{1}$ be the metrics defined by $d(x, y)=\|x-y\|$ and $d_{1}(x, y)=\|x-y\|_{1} .$ Suppose that there exists $K>0$ such that $\|x\| \leq K\|x\|_{1}$ for all $x \in X .$ Let $\left\{x_{n}\right\}$ be a sequence in $X$.
    \begin{enumerate}
        \item If $\left\{x_{n}\right\}$ converges to $x$ in the metric space $\left(X, d_{1}\right)$ then $\left\{x_{n}\right\}$ converges to $x$ in the metric space $(X, d)$
        \item If $\left\{x_{n}\right\}$ is Cauchy in the metric space $\left(X, d_{1}\right)$ then $\left\{x_{n}\right\}$ is Cauchy in the metric space $(X, d)$
    \end{enumerate}
\end{lemma}

\begin{corollary}
    Let $X$ be a vector space and let $\|\cdot\|$ and $\|\cdot\|_{1}$ be equivalent norms on $X .$ Let $d$ and $d_{1}$ be the metrics defined by $d(x, y)=\|x-y\|$ and $d_{1}(x, y)=\|x-y\|_{1}$ Let $\left\{x_{n}\right\}$ be a sequence in $X$.
    \begin{enumerate}
        \item $\left\{x_{n}\right\}$ converges to $x$ in the metric space $(X, d)$ $\iff$ $\left\{x_{n}\right\}$ converges to $x$ in the metric space $\left(X, d_{1}\right)$.
        \item $\left\{x_{n}\right\}$ is Cauchy in the metric space $(X, d)$ $\iff$ $\left\{x_{n}\right\}$ is Cauchy in the metric space $\left(X, d_{1}\right)$.
        \item $(X, d)$ is complete $\iff$ $\left(X, d_{1}\right)$ is complete.
    \end{enumerate}
\end{corollary}

\begin{theorem}
    Let $X$ be a finite-dimensional vector space with norm $\|\cdot\|$ and let $\left\{e_{1}, e_{2}, \ldots, e_{n}\right\}$ be a basis for $X .$ Another norm on $X$ was defined by
    $$
        \left\|\sum_{j=1}^{n} \lambda_{j} e_{j}\right\|_{1}=\left(\sum_{j=1}^{n}\left|\lambda_{j}\right|^{2}\right)^{\frac{1}{2}}
    $$
    The norms $\|\cdot\|$ and $\|\cdot\|_{1}$ are equivalent.
\end{theorem}

\begin{corollary}
    If $\|\cdot\|$ and $\|\cdot\|_{2}$ are any two norms on a finite-dimensional vector space $X$ then they are equivalent.
\end{corollary}

\begin{lemma}
    Let $X$ be a finite-dimensional vector space over $\mathbb{F}$ and let $\left\{e_{1}, e_{2}, \ldots, e_{n}\right\}$ be a basis for $X .$ If $\|\cdot\|_{1}: X \rightarrow \mathbb{R}$ is the norm on $X$ defined by $\left\|\sum_{j=1}^{n} \lambda_{j} e_{j}\right\|_{1}=\left(\sum_{j=1}^{n}\left|\lambda_{j}\right|^{2}\right)^{\frac{1}{2}}$ then $X$ is a complete metric space.
\end{lemma}

\begin{corollary}
    If $\|\cdot\|$ is any norm on a finite-dimensional space $X$ then $X$ is a complete metric
    space.
\end{corollary}

\begin{corollary}
    If $Y$ is a finite-dimensional subspace of a normed vector space $X,$ then $Y$ is closed.
\end{corollary}

\subsection{Banach Spaces}

\par In infinite-dimensional vector space $X$, many methods we discussed in the last section will expire.

\begin{lemma}
    If $X$ is a normed vector space and $S$ is a linear subspace of $X$ then $\bar{S}$ is a linear subspace of $X$.
\end{lemma}


\begin{definition}
    Let $X$ be a normed vector space and let $E$ be any non-empty subset of $X .$ The closed linear span of $E,$ denoted by $\overline{Sp} E,$ is the intersection of all the closed linear subspaces of $X$ which contain $E$.
\end{definition}

\begin{lemma}
    Let $X$ be a normed space and let $E$ be any non-empty subset of $X$.
    \begin{enumerate}
        \item $\overline{\operatorname{Sp}} E$ is a closed linear subspace of $X$ which contains $E$.
        \item $\overline{\mathrm{Sp}} E=\overline{\operatorname{Sp} E},$ that is, $\overline{\operatorname{Sp}} E$ is the closure of $\operatorname{Sp} E$
    \end{enumerate}
\end{lemma}

\begin{theorem}[Riesz’ Lemma]
    Suppose that $X$ is a normed vector space, $Y$ is a closed linear subspace of $X$ such that $Y \neq X$ and $\alpha$ is a real number such that $0<\alpha<1 .$ Then there exists $x_{\alpha} \in X$ such that $\left\|x_{\alpha}\right\|=1$ and $\left\|x_{\alpha}-y\right\|>\alpha$ for all $y \in Y$
\end{theorem}

\begin{theorem}
    If $X$ is an infinite-dimensional normed vector space then neither $D=\{x \in X:$ $\|x\| \leq 1\}$ nor $K=\{x \in X:\|x\|=1\}$ is compact.
\end{theorem}

\begin{definition}[Banach Space]
    A Banach space is a normed vector space which is complete under the metric
    associated with the norm.

    \begin{itemize}
        \item Any finite-dimensional normed vector space is a Banach space.
        \item If $X$ is a compact metric space then $C_{\mathrm{F}}(X)$ is a Banach space.
        \item If $(X, \Sigma, \mu)$ is a measure space then $L^{p}(X)$ is a Banach space for $1 \leq p \leq \infty$.
        \item $\ell^{p}$ is a Banach space for $1 \leq p \leq \infty$.
        \item If $X$ is a Banach space and $Y$ is a linear subspace of $X$ then $Y$ is a Banach space if and only if $Y$ is closed in $X$.
    \end{itemize}
\end{definition}

\begin{theorem}
    Let $X$ be a Banach space and let $\left\{x_{n}\right\}$ be a sequence in $X$. If the series $\sum_{k=1}^{\infty}\left\|x_{k}\right\|$ converges ($\sum_{k=1}^{\infty} x_{k}=\lim _{n \rightarrow \infty} s_{n}$ exists) then the series $\sum_{k=1}^{\infty} x_{k}$ converges.
\end{theorem}

\section{Inner Product Spaces, Hilbert Spaces}

\subsection{Inner Products}
\begin{definition}
    Let $X$ be a vector space. An \textbf{inner product} on $X$ is a function $(\cdot, \cdot):$ $X \times X \rightarrow \mathbb{R}$ such that for all $x, y, z \in X$ and $\alpha, \beta \in \mathbb{R}$
    \begin{itemize}
        \item $(x, x) \geq 0$ ($\in R$ if complex)
        \item $(x, x)=0$ if and only if $x=0$
        \item $(\alpha x+\beta y, z)=\alpha(x, z)+\beta(y, z)$
        \item $(x, y)=(y, x)$. ($(x, y)=\overline{(y, x)}$ for complex vector space)
    \end{itemize}
\end{definition}

\begin{lemma}
    Let $X$ be an inner product space, $x, y \in X $:
    \begin{itemize}
        \item $|(x, y)|^{2} \leq(x, x)(y, y), x, y \in X$
        \item the function $\|\cdot\|: X \rightarrow \mathbb{R}$ defined by $\|x\|=(x, x)^{1 / 2},$ is a norm on $X$.
    \end{itemize}
\end{lemma}

\begin{theorem}
    To prove a norm:
    \begin{itemize}
        \item $ \|x+y\|^{2}+\|x-y\|^{2}=2\left(\|x\|^{2}+\|y\|^{2}\right) $ (the parallelogram rule)
        \item $4(x, y)=\|x+y\|^{2}-\|x-y\|^{2}+i\|x+i y\|^{2}-i\|x-i y\|^{2}$ (the polarization identity)
    \end{itemize}
\end{theorem}

\begin{lemma}
    If $\lim _{n \rightarrow \infty} x_{n}=x, \lim _{n \rightarrow \infty} y_{n}=y$, then $\lim _{n \rightarrow \infty}\left(x_{n}, y_{n}\right)=(x, y)$
\end{lemma}

\subsection{Orthogonality}

\begin{definition}
    Orthogonal: $(x, y)=0$, Orthonormal basis:$\left\{e_{1}, \ldots, e_{k}\right\}$ that $x=\sum_{n=1}^{k}\left(x, e_{n}\right) e_{n}$
\end{definition}

\begin{lemma}
    $Gram-Schmidt$ algorithm: $b_{k+1}=v_{k+1}-\sum_{n=1}^{k}\left(v_{k+1}, e_{n}\right) e_{n}, \quad e_{k+1}=\frac{b_{k+1}}{\left\|b_{k+1}\right\|}$
\end{lemma}

\begin{theorem}
    Orthonormal basis then: $\left\|\sum_{n=1}^{k} \alpha_{n} e_{n}\right\|^{2}=\sum_{n=1}^{k}\left|\alpha_{n}\right|^{2}$
\end{theorem}

\begin{definition}
    An inner product space which is complete with respect to the metric associated with the norm induced by the inner product is called a \textbf{Hilbert space}. (eg. every finite-dimensional inner product space, $L2$ and $\ell^{2}$ with standard inner product)
\end{definition}

\begin{lemma}
    $Y \subset \text{Hilbert space } \mathcal{H}$ is a linear subspace, then $Y$ is a Hilbert space if and only if $Y$ is closed.
\end{lemma}

\subsection{Orthogonal Complements}

\begin{definition}
    $X$ be an inner product space and let $A$ be a subset of $X .$ The orthogonal complement of $A$ is the set
    $$
        A^{\perp}=\{x \in X:(x, a)=0 \text { for all } a \in A\}
    $$
\end{definition}

\begin{lemma}
    If $X$ is an inner product space and $A \subset X$ then:
    \begin{itemize}
        \item $0 \in A^{\perp}$.
        \item If $0 \in A$ then $A \cap A^{\perp}=\{0\},$ otherwise $A \cap A^{\perp}=\emptyset$
        \item $\{0\}^{\perp}=X ; X^{\perp}=\{0\}$
        \item If $A$ contains an open ball $\bar{B}_{a}(r),$ for some $a \in \bar{X}$ and some positive $r>\overline{0}$ then $A^{\perp}=\{0\} ;$ in particular, if $A$ is a non-empty open set then $A^{\perp}=\{0\}$
        \item If $B \subset A$ then $A^{\perp} \subset B^{\perp}$.
        \item $A^{\perp}$ is a closed linear subspace of $X$.
        \item $A \subset\left(A^{\perp}\right)^{\perp}$.
    \end{itemize}
\end{lemma}

\begin{lemma}
    Let $Y$ be a linear subspace of an inner product space $X .$ Then
    $$
        x \in Y^{\perp} \Longleftrightarrow\|x-y\| \geq\|x\|, \quad \forall y \in Y
    $$
\end{lemma}

\begin{definition}
    A subset $A$ of a vector space $X$ is \textbf{convex} if, for all $x, y \in A$ and all $\lambda \in[0,1]$ $\lambda x+(1-\lambda) y \in A$
\end{definition}

\begin{theorem}
    $A$ non-empty, closed, convex subset of a Hilbert space $\mathcal{H}$ and $p \in \mathcal{H}$. There exists a unique $q \in A$ such that
    $$
        \|p-q\|=\inf \{\|p-a\|: a \in A\}
    $$
\end{theorem}

\begin{theorem}(Orthogonal decomposition)
    $Y$ a closed linear subspace of a Hilbert space $\mathcal{H}$. For any $x \in \mathcal{H},$ there exists a unique $y \in Y$ and $z \in Y^{\perp}$ such that $x=y+z .$ Also $,\|x\|^{2}=\|y\|^{2}+\|z\|^{2}$.
\end{theorem}

\begin{lemma}
    If $Y$ is a closed linear subspace of a Hilbert space $\mathcal{H}$ then $Y^{\perp \perp}=Y$. If $Y$ is any linear subspace of a Hilbert space $\mathcal{H}$ then $Y^{\perp \perp}=\bar{Y}$.
\end{lemma}

\subsection{Orthonormal Bases in Infinite Dimensions}

\begin{definition}[Orthonormal Sequence]
    Let $X$ be an inner product space. A sequence $\left\{e_{n}\right\} \subset X$ is  an orthonormal sequence if $\left\|e_{n}\right\|=1$ for all $n \in \mathbb{N},$ and $\left(e_{m}, e_{n}\right)=0$ for all $m, n \in \mathbb{N}$ with $m \neq n$
\end{definition}

\begin{theorem}
    Any infinite-dimensional inner product space $X$ contains an orthonormal sequence.
\end{theorem}

\begin{lemma}[Bessel’s Inequality]
    Let $X$ be an inner product space and let $\left\{e_{n}\right\}$ be an orthonormal sequence in $X .$ For any $x \in X$ the (real) series $\sum_{n=1}^{\infty}\left|\left(x, e_{n}\right)\right|^{2}$ converges and
    $$\sum_{n=1}^{\infty}\left|\left(x, e_{n}\right)\right|^{2} \leq\|x\|^{2}$$
\end{lemma}

\begin{theorem}
    $\left\{e_{n}\right\}$ be an orthonormal sequence in $\mathcal{H}$, $\left\{\alpha_{n}\right\}$ be a sequence in $\mathbb{F} .$ Then the series $\sum_{n=1}^{\infty} \alpha_{n} e_{n}$ converges if and only if $\sum_{n=1}^{\infty}\left|\alpha_{n}\right|^{2}<\infty$(if and only if the sequence $\left\{\alpha_{n}\right\} \in \ell^{2}$.) If this holds, then
    $$
        \left\|\sum_{n=1}^{\infty} \alpha_{n} e_{n}\right\|^{2}=\sum_{n=1}^{\infty}\left|\alpha_{n}\right|^{2}
    $$
\end{theorem}

\begin{corollary}
    Let $\left\{e_{n}\right\}$ be an orthonormal sequence in $\mathcal{H} .$ For any $x \in \mathcal{H}$ the series $\sum_{n=1}^{\infty}\left(x, e_{n}\right) e_{n}$ converges.
\end{corollary}

\begin{theorem}
    Let $\left\{e_{n}\right\}$ be an orthonormal sequence in $\mathcal{H}$, the following conditions are equivalent:
    \begin{itemize}
        \item $\left\{e_{n}: n \in \mathbb{N}\right\}^{\perp}=\{0\}$
        \item $\overline{\operatorname{Sp}}\left\{e_{n}: n \in \mathbb{N}\right\}=\mathcal{H}$
        \item $\|x\|^{2}=\sum_{n=1}^{\infty}\left|\left(x, e_{n}\right)\right|^{2}$ for all $x \in \mathcal{H}$
        \item $x=\sum_{n=1}^{\infty}\left(x, e_{n}\right) e_{n}$ for all $x \in \mathcal{H}$.
    \end{itemize}
    Called orthogonal basis if any holds.
\end{theorem}

\begin{theorem}
    \begin{enumerate}
        \item Finite dimensional normed vector spaces are separable.
        \item An infinite-dimensional Hilbert space $\mathcal{H}$ is separable if and only if it has an orthonormal basis.
    \end{enumerate}
\end{theorem}

\subsection{Fourier Series}

\begin{theorem}
    The set of functions $ C=\left\{c_{0}(x)=(1 / \pi)^{1 / 2}, c_{n}(x)=(2 / \pi)^{1 / 2} \cos n x: n \in \mathbb{N}\right\} $ is an orthonormal basis in $L^{2}[0, \pi]$.
\end{theorem}

\begin{corollary}
    The space $L^{2}[0, \pi]$ is separable.
\end{corollary}

\begin{corollary}
    The sets of functions
    $$
        \begin{array}{l}
            E=\left\{e_{n}(x)=(2 \pi)^{-1 / 2} e^{i n x}: n \in \mathbb{Z}\right\} \\
            F=\left\{2^{-1 / 2} c_{0}, 2^{-1 / 2} c_{n}, 2^{-1 / 2} s_{n}: n \in \mathbb{N}\right\}
        \end{array}
    $$
    are orthonormal bases in the space $L_{\mathrm{C}}^{2}[-\pi, \pi] .$ The set $F$ is also an orthonormal basis in the space $L_{\mathbb{R}}^{2}[-\pi, \pi]$ (the set $E$ is clearly not appropriate for the space $L_{\mathbb{R}}^{2}[-\pi, \pi]$ since the functions in $E$ are complex $)$
\end{corollary}

\section{Linear Operators}
\subsection{Continuous Linear Transformations}

\begin{lemma}
    Let $X$ and $Y$ be normed linear spaces and let $T: X \rightarrow Y$ be a linear transformation. The following are equivalent:
    \begin{enumerate}
        \item $T$ is uniformly continuous;
        \item $T$ is continuous;
        \item $T$ is continuous at 0
        \item there exists a positive real number $k$ such that $\|T(x)\| \leq k$ whenever $x \in X$ and $\|x\| \leq 1$
        \item there exists a positive real number $k$ such that $\|T(x)\| \leq k\|x\|$ for all $x \in X$
    \end{enumerate}
\end{lemma}

\begin{lemma}
    If $\left\{c_{n}\right\} \in \ell^{\infty}$ and $\left\{x_{n}\right\} \in \ell^{p},$ where $1 \leq p<\infty,$ then $\left\{c_{n} x_{n}\right\} \in \ell^{p}$ and $ \sum_{n=1}^{\infty}\left|c_{n} x_{n}\right|^{p} \leq\left\|\left\{c_{n}\right\}\right\|_{\infty}^{p} \sum_{n=1}^{\infty}\left|x_{n}\right|^{p}$
\end{lemma}

\begin{definition}[Bounded]
    Let $X$ and $Y$ be normed linear spaces and let $T: X \rightarrow Y$ be a linear transformation. $T$ is said to be \textbf{bounded} if there exists a positive real number $k$ such that $\|T(x)\| \leq k\|x\|$ for all $x \in X$
\end{definition}

\begin{theorem}
    Let $X$ be a finite-dimensional normed space, let $Y$ be any normed linear space and let $T: X \rightarrow Y$ be a linear transformation. Then $T$ is continuous.
\end{theorem}

\begin{lemma}
    If $X$ and $Y$ are normed linear spaces and $T: X \rightarrow Y$ is a continuous linear transformation then Ker $(T)$ is closed.
\end{lemma}

\begin{definition}[Graph]
    If $X$ and $Y$ are normed spaces and $T: X \rightarrow Y$ is a linear transformation, the \textbf{graph} of $T$ is the linear subspace $\mathcal{G}(T)$ of $X \times Y$ defined by $\mathcal{G}(T)=\{(x, T x): x \in X\}$
\end{definition}

\begin{lemma}
    If $X$ and $Y$ are normed spaces and $T: X \rightarrow Y$ is a continuous linear transformation then $\mathcal{G}(T)$ is closed.
\end{lemma}

\begin{lemma}
    Let $X$ and $Y$ be normed linear spaces and let $S, T \in B(X, Y)$ with $\|S(x)\| \leq$ $k_{1}\|x\|$ and $\|T(x)\| \leq k_{2}\|x\|$ for all $x \in X .$ Let $\lambda \in \mathbb{F} .$ Then
    \begin{enumerate}
        \item $\|(S+T)(x)\| \leq\left(k_{1}+k_{2}\right)\|x\|$ for all $x \in X$
        \item $\|(\lambda S)(x)\| \leq|\lambda| k_{1}\|x\|$ for all $x \in X$
        \item $B(X, Y)$ is a linear subspace of $L(X, Y)$ and so $B(X, Y)$ is a vector space.
    \end{enumerate}
\end{lemma}

\subsection{The Norm of a Bounded Linear Operator}

\begin{definition}[Norm]
    Let $X$ and $Y$ be normed spaces. If $\|\cdot\|: B(X, Y) \rightarrow \mathbb{R}$ is defined by $ \|T\|=\sup \{\|T(x)\|:\|x\| \leq 1\} $ then $\|\cdot\|$ is a norm on $B(X, Y)$.
\end{definition}

\begin{definition}[Norm of Matrix]
    Let $\mathbb{F}^{p}$ have the standard norm and let $A$ be a $m \times n$ matrix with entries in $F$. If $T: \mathbb{F}^{n} \rightarrow \mathbb{F}^{m}$ is the bounded linear transformation defined by $T(x)=A x$ then the norm of the matrix $A$ is defined by $\|A\|=\|T\|$.
\end{definition}

\begin{theorem}
    Let $X$ be a normed linear space and let $W$ be a dense subspace of $X .$ Let $Y$ be a Banach space and let $S \in B(W, Y)$.
    \begin{enumerate}
        \item If $x \in X$ and $\left\{x_{n}\right\}$ and $\left\{y_{n}\right\}$ are sequences in $W$ such that $\lim _{n \rightarrow \infty} x_{n}=$ $\lim _{n \rightarrow \infty} y_{n}=x$ then $\left\{S\left(x_{n}\right)\right\}$ and $\left\{S\left(y_{n}\right)\right\}$ both converge and $\lim _{n \rightarrow \infty} S\left(x_{n}\right) = \lim _{n \rightarrow \infty} S\left(y_{n}\right)
              $
        \item There exists $T \in B(X, Y)$ such that $\|T\|=\|S\|$ and $T x=S x$ for all $x \in W$
    \end{enumerate}
\end{theorem}

\begin{definition}[isometry]
    Let $X$ and $Y$ be normed linear spaces and let $T \in L(X, Y) .$ If $\|T(x)\|=\|x\|$ for all $x \in X$ then $T$ is called an isometry.
\end{definition}

\begin{lemma}
    Let $X$ and $Y$ be normed linear spaces and let $T \in L(X, Y) .$ If $T$ is an isometry then $T$ is bounded and $\|T\|=1$.
\end{lemma}

\begin{definition}
    If $X$ and $Y$ are nurmed linear spaces and $T$ is an isometry from $X$ onto $Y$ then $T$ is called an \textbf{isometric isomorphism} and $X$ and $Y$ are called \textbf{isometrically isomorphic}.
\end{definition}

\begin{theorem}
    Let $\mathcal{H}$ be an infinite-dimensional Hilbert space over $\mathbb{F}$ with an orthonormal basis $\left\{e_{n}\right\} .$ Then there is an isometry $T$ of $\mathcal{H}$ onto $\ell_{\mathrm{F}}^{2}$ such that $T\left(e_{n}\right)=\widetilde{e}_{n}$ for all $n \in \mathbb{N}$
\end{theorem}

\begin{corollary}
    Any infinite-dimensional, separable Hilbert space $\mathcal{H}$ over $\mathbb{F}$ is isometrically isomorphic to $\ell_{\mathrm{F}}^{2}$
\end{corollary}

\subsection{The Space $B(X, Y)$}

\begin{theorem}
    If $X$ is a normed linear space and $Y$ is a Banach space then $B(X, Y)$ is a Banach space.
\end{theorem}

\begin{definition}[dual space]
    Let $X$ be a normed space. Linear transformations from $X$ to $\mathbb{F}$ are called linear functionals. The space $B(X, \mathbb{F})$ is called the \textbf{dual space} of $X$ and denoted $X^{\prime}$
\end{definition}

\begin{corollary}
    If $X$ is a normed vector space then $X^{\prime}$ is a Banach space.
\end{corollary}

\begin{lemma}
    If $X, Y$ and $Z$ are normed linear spaces and $T \in B(X, Y)$ and $S \in B(Y, Z)$ then $S \circ T \in B(X, Z)$ and $\|S \circ T\| \leq\|S\|\|T\|$
\end{lemma}

\begin{definition}[product]
    Let $X, Y$ and $Z$ be normed linear spaces and $T \in B(X, Y)$ and $S \in B(Y, Z)$ The composition $S \circ T$ of $S$ and $T$ will be denoted by $S T$ and called the product of $S$ and $T$
\end{definition}

\begin{lemma}
    Let $X$ be a normed linear space.
    \begin{enumerate}
        \item $B(X)$ is an algebra with identity and hence a ring with identity.
        \item If $\left\{T_{n}\right\}$ and $\left\{S_{n}\right\}$ are sequences in $B(X)$ such that $\lim _{n \rightarrow \infty} T_{n}=T$ and $\lim _{n \rightarrow \infty} S_{n}=S$ then $\lim _{n \rightarrow \infty} S_{n} T_{n}=S T$
    \end{enumerate}
\end{lemma}

\begin{lemma}
    Let $X$ be a normed linear space and let $T \in B(X) .$ If $p$ and $q$ are polynomials and $\lambda, \mu \in \mathbb{C}$ then
    \begin{enumerate}
        \item $(\lambda p+\mu q)(T)=\lambda p(T)+\mu q(T)$
        \item $(p q)(T)=p(T) q(T)$
    \end{enumerate}
\end{lemma}

\subsection{Inverses of Operators }

\begin{definition}[invertable]
    Let $X, Y$ be normed linear spaces. An operator $T \in B(X, Y)$ is said to be \textbf{invertible} if there exists $S \in B(Y, X)$ such that $S T=I_{X}, T S=I_{Y},$ in which case $S$ is the inverse of $T$ and is denoted by $T^{-1}$.
\end{definition}

\begin{lemma}
    If $X, Y, Z$ are normed linear spaces and $T_{1} \in B(X, Y), T_{2} \in B(Y, Z)$ are invertible, then:
    \begin{enumerate}
        \item $T_{1}^{-1}$ is invertible with inverse $T_{1}$
        \item $T_{2} T_{1}$ is invertible with inverse $T_{1}^{-1} T_{2}^{-1}$
    \end{enumerate}
\end{lemma}

\begin{definition}[isomorphism]
    Let $X, Y$ be normed linear spaces. If an invertible operator $T \in B(X, Y)$ exists then $X, Y$ are isomorphic, and $T$ is an isomorphism between $X$ and $Y$ .
\end{definition}

\begin{lemma}
    If the normed linear spaces $X, Y,$ are isomorphic, then:
    \begin{enumerate}
        \item  $\operatorname{dim} X<\infty$ if and only if $\operatorname{dim} Y<\infty,$ in which case $\operatorname{dim} X=\operatorname{dim} Y$
        \item $X$ is separable if and only if $Y$ is separable;
        \item $X$ is complete (i.e., Banach) if and only if $Y$ is complete (i.e., Banach).
    \end{enumerate}
\end{lemma}

\begin{theorem}
    Let $X$ be a Banach space. If $T \in B(X)$ is an operator with $\|T\|<1$ then $I-T$ is invertible and the inverse is given by $(I-T)^{-1}=\sum_{n=0}^{\infty} T^{n}$ (called $Neumann \ series$)
\end{theorem}

\begin{corollary}
    Let $X, Y$ be Banach spaces. The set $\mathcal{A}$ of invertible operators in $B(X, Y)$ is open.
\end{corollary}

\begin{theorem}[Open mapping]
    Suppose that $X$ and $Y$ are Banach spaces and $T \in B(X, Y)$ is surjective. Let
    $$
        L=\{T(x): x \in X \text { and }\|x\| \leq 1\}
    $$
    with closure $\bar{L}$. Then:
    \begin{enumerate}
        \item there exists $r>0$ such that $\{y \in Y:\|y\| \leq r\} \subseteq \bar{L} ;$
        \item $\left\{y \in Y:\|y\| \leq \frac{r}{2}\right\} \subseteq L$
        \item if, in addition, $T$ is one-to-one then $T$ is invertible.
    \end{enumerate}
\end{theorem}

\begin{corollary}[closed graph]
    If $X$ and $Y$ are Banach space and $T$ is a linear transformation from $X$ into $Y$ such that $\mathcal{G}(T),$ the graph of $T,$ is closed, then $T$ is continuous.
\end{corollary}

\begin{corollary}[Banach's Isomorphism Theorem]
    If $X, Y$ are Banach spaces and $T \in B(X, Y)$ is bijective, then $T$ is invertible
\end{corollary}

\begin{lemma}
    If $X, Y$ are normed linear spaces and $T \in B(X, Y)$ is invertible then, for all $x \in X$
    $$
        \|T x\| \geq\left\|T^{-1}\right\|^{-1}\|x\|
    $$
\end{lemma}

\begin{lemma}
    Suppose that $X$ is a Banach space, $Y$ is a normed space and $T \in B(X, Y) .$ If there exists $\alpha>0$ such that $\|T x\| \geq \alpha\|x\|$ for all $x \in X,$ then $\operatorname{Im}(T)$ is closed.
\end{lemma}

\begin{theorem}
    Suppose that $X, Y$ are Banach spaces, and $T \in B(X, Y) .$ The following are equivalent:
    \begin{enumerate}
        \item $T$ is invertible
        \item $\operatorname{Im}(T)$ is dense in $Y$ and there exists $\alpha>0$ such that $\|T x\| \geq \alpha\|x\|$ for all $x \in X$
    \end{enumerate}
\end{theorem}

\begin{corollary}
    Suppose that $X, Y$ are Banach spaces, and $T \in B(X, Y) .$ The operator $T$ is not invertible if and only if $\operatorname{Im}(T)$ is not dense in $Y$ or there exists a sequence $\left\{x_{n}\right\}$ in $X$ with $\left\|x_{n}\right\|=1$ for all $n \in \mathbb{N}$ but $\lim _{n \rightarrow \infty} T x_{n}=0$
\end{corollary}

\begin{theorem}[Uniform Boundedness Principle]
    Let $U, X$ be Banach spaces. Suppose that $S$ is a non-empty set and, for each $s \in S, T_{s} \in B(U, X) .$ If, for each $u \in U,$ the set $\left\{\left\|T_{s}(u)\right\|: s \in S\right\}$ is bounded then the set $\left\{\left\|T_{s}\right\|: s \in S\right\}$ is bounded.
\end{theorem}

\begin{corollary}
    Let $U, X$ be Banach spaces and $T_{n} \in B(U, X), n=1,2, \ldots$ Suppose that, $\lim _{n \rightarrow \infty} T_{n} u$ exists, for each $u \in U,$ and define $T u=\lim _{n \rightarrow \infty} T_{n} u .$ Then $T \in B(U, X)$
\end{corollary}

\section{Duality and the Hahn–Banach Theorem}

\subsection{Dual Spaces}

\begin{theorem}
    If $X$ is a finite dimensional normed linear space with basis $\left\{v_{1}, v_{2}, \ldots, v_{n}\right\}$ then $X^{\prime}$ has a basis $\left\{f_{1}, f_{2}, \ldots, f_{n}\right\}$ such that $f_{j}\left(v_{k}\right)=\delta_{j k},$ for $1 \leq j, k \leq n .$ In particular, $\operatorname{dim} X^{\prime}=\operatorname{dim} X$
\end{theorem}

\begin{theorem}{(Riesz–Frechet Theorem}
    Let $\mathcal{H}$ be a Hilbert space and let $f \in \mathcal{H}^{\prime} .$ Then there is a unique $y \in \mathcal{H}$ such that $f(x)=f_{y}(x)=(x, y)$ for all $x \in \mathcal{H} .$ Moreover $\|f\|=\|y\|$
\end{theorem}

\begin{theorem}
    Let $\mathcal{H}$ be a Hilbert space, and define $T_{\mathcal{H}}: \mathcal{H} \rightarrow \mathcal{H}^{\prime}$ by $T_{\mathcal{H}} y=f_{y}, y \in \mathcal{H} .$ Then
    $T_{\mathcal{H}}$ is a bijection, and for all $\alpha, \beta \in \mathbb{F}, y, z \in \mathcal{H}:$
    \begin{enumerate}
        \item $T_{\mathcal{H}}(\alpha y+\beta z)=\bar{\alpha} T_{\mathcal{H}} y+\bar{\beta} T_{\mathcal{H}} z$
        \item $\left\|T_{\mathcal{H}} y\right\|=\|y\|$.
    \end{enumerate}
\end{theorem}

\begin{lemma}
    For an arbitrary integer $k \geq 1,$ let $\mathcal{S}_{k} \subset \ell^{\infty}$ consist of sequences of the form $x=\left(x_{1}, \ldots, x_{k}, 0,0, \ldots\right),$ and let $\mathcal{S}=\bigcup_{k \geq 1} \mathcal{S}_{k} .$ The set $\mathcal{S}$ is dense in $\ell^{p}$ $1 \leq p<\infty,$ but not in $\ell^{\infty}$
\end{lemma}

\begin{theorem}
    Suppose that $1 \leq p<\infty .$ If $p>1,$ let $q=p /(p-1),$ while if $p=1,$ let $q=\infty$
    \begin{enumerate}
        \item If $a=\left\{a_{n}\right\} \in \ell^{q}$ then, for any $x=\left\{x_{n}\right\} \in \ell^{p},$ the sequence $\left\{a_{n} x_{n}\right\} \in \ell^{1}$, and $f_{a}(x)=\sum_{n=1}^{\infty} a_{n} x_{n}, \quad x \in \ell^{p}$
              defines a linear functional $f_{a} \in\left(\ell^{p}\right)^{\prime},$ with $\left\|f_{a}\right\|=\|a\|_{q}$
        \item If $f \in\left(\ell^{p}\right)^{\prime}$ then there exists a unique $a \in \ell^{q}$ such that $f=f_{a}$.
        \item By parts (a) and (b), the function $T_{p}: \ell^{q} \rightarrow\left(\ell^{p}\right)^{\prime}$ defined by $T_{p}(a)=f_{a}$ $a \in \ell^{q},$ is a linear isometric isomorphism.
    \end{enumerate}
\end{theorem}

\subsection{Hahn–Banach Theorem}

\begin{definition}[extension]
    Let $X$ be a vector space, $W$ a linear subspace of $X,$ and $f_{W}$ a linear functional on $W$. A linear functional $f_{X}$ on $X$ is an \textbf{extension} of $f_{W}$ if $f_{X}(w)=f_{W}(w)$ for all $w \in W$.
\end{definition}

\begin{definition}[sublinear functional]
    Let $X$ be a real vector space. A sublinear functional on $X$ is a function $p: X \rightarrow$ $\mathbb{R}$ such that:
    \begin{enumerate}
        \item $p(x+y) \leq p(x)+p(y), \quad x, y \in X$
        \item $p(\alpha x)=\alpha p(x), \quad x \in X, \alpha \geq 0$
    \end{enumerate}
\end{definition}

\begin{definition}[seminorm]
    Let $X$ be a real or complex vector space. A seminorm on $X$ is a real-valued function $p: X \rightarrow \mathbb{R}$ such that:
    \begin{enumerate}
        \item $p(x+y) \leq p(x)+p(y), x, y \in X$
        \item $p(\alpha x)=|\alpha| p(x), x \in X, \alpha \in \mathbb{F}$.
    \end{enumerate}
\end{definition}

\begin{theorem}[Hahn-Banach Theorem]
    Let $X$ be a real vector space, with a sublinear functional $p$ defined on $X$. Suppose that $W$ is a linear subspace of $X$ and $f_{W}$ is a linear functional on $W$ satisfying
    $$
        f_{W}(w) \leq p(w), \quad w \in W
    $$
    Then $f_{W}$ has an extension $f_{X}$ on $X$ such that
    $$
        f_{X}(x) \leq p(x), \quad x \in X
    $$
\end{theorem}

\begin{lemma}\label{lf}
    If $g$ is a linear functional on $V$ then there exists a unique, real-valued, linear functional $g_{R}$ on $V_{R}$ such that $g(v)=g_{R}(v)-i g_{R}(i v), \quad v \in V$
\end{lemma}

\begin{lemma}
    Let $X$ be a complex vector space and let $p$ be a seminorm on $X .$ Suppose that $W$ is a linear subspace of $X$ and $f_{W}$ is a linear functional on $W$ satisfying
    $$
        \left|f_{W}(w)\right| \leq p(w), \quad w \in W
    $$
    Suppose that $f_{W, R},$ the real functional on $W_{R}$ obtained by applying \ref{lf} to $f_{W},$ has an extension $f_{X, R}$ on $X_{R},$ satisfying
    $$
        \left|f_{X, R}(x)\right| \leq p(x), \quad x \in X_{R}
    $$
    Then $f_{W}$ has an extension $f_{X}$ on $X$ such that
    $$
        \left|f_{X}(x)\right| \leq p(x), \quad x \in X
    $$
\end{lemma}

\begin{theorem}[Hahn-Banach Theorem]
    Let $X$ be a real or complex vector space and let $p$ be a seminorm on $X .$ Suppose that $W$ is a linear subspace of $X$ and $f_{W}$ is a linear functional on $W$ satisfying
    $$
        \left|f_{W}(w)\right| \leq p(w), \quad w \in W
    $$
    Then $f_{W}$ has an extension $f_{X}$ on $X$ such that
    $$
        \left|f_{X}(x)\right| \leq p(x), \quad x \in X
    $$
\end{theorem}


\subsection{The Hahn–Banach Theorem in Normed Spaces}



























\end{document}