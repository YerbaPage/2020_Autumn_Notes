\documentclass[10pt]{paper}
% \usepackage[utf8]{inputenc}
\usepackage{fancyhdr}
\pagestyle{plain}
\usepackage[english]{babel}
\usepackage{hyperref}
\usepackage{amstext}
\usepackage{amssymb}
\usepackage{amsmath}
\usepackage[a4paper,scale=0.9]{geometry}
\usepackage{xcolor}
\definecolor{winered}{rgb}{0.5,0,0}
\hypersetup{
  pdfborder={0 0 0},
  colorlinks=true,
  linkcolor={winered},
  urlcolor={winered},
  filecolor={winered},
  citecolor={winered},
  linktoc=all,
}


\newtheorem{theorem}{Theorem}[section]
\newtheorem{definition}{Def}[section]
\newtheorem{corollary}{Corollary}[section]
\newtheorem{lemma}[theorem]{Lemma}

\title{Linear Functional Analysis Notes}
\date{\today}
\author{Yuling Shi}
\begin{document}
\maketitle
\tableofcontents
% \footnotesize
\section{Preliminaries}

\begin{theorem}
    Suppose that $(M, d)$ is a metric space and $A \subset M .$ Then:
    \begin{enumerate}
        \item if $A$ is complete then it is closed;
        \item if $M$ is complete then $A$ is complete if and only if it is closed;
        \item if $A$ is compact then it is closed and bounded;
        \item (Bolzano-Weierstrass theorem) every closed, bounded subset of $\mathbb{F}^{k}$ is compact.
    \end{enumerate}
\end{theorem}

\begin{definition}[counting measure]
    We define counting measure
    $$\mu(A)=\left\{\begin{array}{ll}
            n      & \text{if} A \text { has exactly } n \text { elements } \\
            \infty & \text { otherwise }
        \end{array}\right.$$
\end{definition}

\begin{definition}[essential supremum]
    Suppose that $f$ is a measurable function and there exists a number $b$ such that $f(x) \leq b a.e.$ Then we can define the \textbf{essential supremum} of $f$ to be $$ess \ sup \  f=\inf \{b: f(x) \leq b \  a.e. \}$$
\end{definition}

\begin{definition}
    We define an equivalence relation $\equiv$ on $\mathcal{L}^{1}(X)$ by $$f \equiv g \Longleftrightarrow f(x)=g(x) \text { for a.e. } x \in X$$ This relation partitions the set $\mathcal{L}^{1}(X)$ into a space of equivalence classes, which we will denote by $L^{1}(X)$.
\end{definition}

\begin{definition}
    Define the spaces$$\begin{array}{l}
            \mathcal{L}^{p}(X)=\left\{f: f \text { is measurable and }\left(\int_{X}|f|^{p} d \mu\right)^{1 / p}<\infty\right\}, \quad 1 \leq p<\infty \\
            \mathcal{L}^{\infty}(X)=\{f: f \text { is measurable and ess sup }|f|<\infty\}
        \end{array}$$
    We also define the corresponding sets $L^{P}(X)$ by identifying functions in $\mathcal{L}^{p}(X)$ which are $a.e.$ equal. In practice, we simply refer to such $\mathcal{L}$ rather than classes themselves.
\end{definition}

\begin{theorem}[Two useful inequalities]
    Minkowski's inequality $($ for $1 \leq p<\infty)$
    $$
        \begin{array}{c}
            \left(\int_{X}|f+g|^{p} d \mu\right)^{1 / p} \leq\left(\int_{X}|f|^{p} d \mu\right)^{1 / p}+\left(\int_{X}|g|^{p} d \mu\right)^{1 / p} \\
            ess\  \sup |f+g| \leq ess\  \sup |f|+ess\  \sup |g|
        \end{array}
    $$
    Hölder's inequality (for $1<p<\infty$ and $p^{-1}+q^{-1}=1$ ):
    $$
        \begin{array}{l}
            \int_{X}|f g| d \mu \leq\left(\int_{X}|f|^{p} d \mu\right)^{1 / p}\left(\int_{X}|g|^{q} d \mu\right)^{1 / q} \\
            \int_{X}|f g| d \mu \leq ess\  \sup |f| \int_{X}|g| d \mu
        \end{array}
    $$
\end{theorem}


\begin{theorem}
    Suppose that $1 \leq p \leq \infty .$ Then the metric space $L^{P}(X)$ is complete. In particular, the sequence space $\ell^{P}$ is complete.
\end{theorem}

\begin{theorem}
    Suppose that $[a, b]$ is a bounded interval and $1 \leq p<\infty .$ Then the set $C[a, b]$ is dense in $L^{p}[a, b]$
\end{theorem}



\section{Normed Spaces}
\subsection{Examples of Normed Spaces}
\begin{definition}[Norm]
    Let $X$ be a vector space over $\mathbb{F}$. A norm on $X$ is a function $\|\cdot\|: X \rightarrow \mathbb{R}$ such that for all $x, y, \in X$ and $\alpha \in \mathbb{F}$
    \begin{enumerate}
        \item $\|x\| \geq 0$
        \item $\|x\|=0$ if and only if $x=0$
        \item $\|\alpha x\|=\|\alpha\|\|x\|$
        \item $\|x+y\| \leq\|x\|+\|y\|$
    \end{enumerate}
    A vector space X on which there is a norm is called a normed vector space or just a normed space. If $X$ is a normed space, a unit vector in $X$ is a vector $x$ such that $\|x\|=1$
\end{definition}

\begin{lemma}
    Let $X$ be a vector space with norm $\|\cdot\| .$ If $d: X \times X \rightarrow \mathbb{R}$ is defined by $d(x, y)=\|x-y\|$ then $(X, d)$ is a metric space.
\end{lemma}

\begin{definition}[Metric]
    If $X$ is a vector space with norm $\|\cdot\|$ and $d$ is the metric defined by $d(x, y)=$ $\|x-y\|$ then $d$ is called the metric associated with $\|\cdot\| \cdot$.
\end{definition}

\begin{definition}[Convergence]
    We are trying to define convergence in our normed space. We will have strong convergence as $\lim _{n \rightarrow \infty}\left\|x_{n} - x\right\|=0$ which is also written as $\lim _{n \rightarrow \infty}x_{n}=x$. And we also have weak convergence as $\lim _{n \rightarrow \infty}\left\|x_{n}\right\|=\|x\|$.
\end{definition}

\begin{theorem}
    Let $X$ be a vector space over $\mathbb{F}$ with norm $\|\cdot\| .$ Let $\left\{x_{n}\right\}$ and $\left\{y_{n}\right\}$ be sequences in $X$ which converge to $x, y$ in $X$ respectively and let $\left\{\alpha_{n}\right\}$ be a sequence in $\mathbb{F}$ which converges to $\alpha$ in $\mathbb{F}$. Then:
    \begin{enumerate}
        \item $|\|x\|-\|y\|| \leq\|x-y\|$
        \item $\lim _{n \rightarrow \infty}\left\|x_{n}\right\|=\|x\|$
        \item $\lim _{n \rightarrow \infty}\left(x_{n}+y_{n}\right)=x+y$
        \item $\lim _{n \rightarrow \infty} \alpha_{n} x_{n}=\alpha x$
    \end{enumerate}

\end{theorem}

\subsection{Finite-dimensional Normed Spaces}
\begin{definition}[Equivalence]
    Let $X$ be a vector space and let $\|\cdot\|_{1}$ and $\|\cdot\|_{2}$ be two norms on $X .$ The norm $\|\cdot\|_{2}$ is equivalent to the norm $\|\cdot\|_{1}$ if there exists $M, m>0$ such that for all $x \in X$
    $$
        m\|x\|_{1} \leq\|x\|_{2} \leq M\|x\|_{1}
    $$
\end{definition}

\begin{lemma}
    Let $X$ be a vector space and let $\|\cdot\|_{1},\|\cdot\|_{2}$ and $\|\cdot\|_{3}$ be three norms on $X$. Let $\|\cdot\|_{2}$ be equivalent to $\|\cdot\|_{1}$ and let $\|\cdot\|_{3}$ be equivalent to $\|\cdot\|_{2}$
    \begin{enumerate}
        \item $\|\cdot\|_{1}$ is equivalent to $\|\cdot\|_{2}$ (exchanging position)
        \item $\|\cdot\|_{3}$ is equivalent to $\|\cdot\|_{1}$ (passing)
    \end{enumerate}
\end{lemma}

\begin{lemma}
    Let $X$ be a vector space and let $\|\cdot\|$ and $\|\cdot\|_{1}$ be norms on $X .$ Let $d$ and $d_{1}$ be the metrics defined by $d(x, y)=\|x-y\|$ and $d_{1}(x, y)=\|x-y\|_{1} .$ Suppose that there exists $K>0$ such that $\|x\| \leq K\|x\|_{1}$ for all $x \in X .$ Let $\left\{x_{n}\right\}$ be a sequence in $X$.
    \begin{enumerate}
        \item If $\left\{x_{n}\right\}$ converges to $x$ in the metric space $\left(X, d_{1}\right)$ then $\left\{x_{n}\right\}$ converges to $x$ in the metric space $(X, d)$
        \item If $\left\{x_{n}\right\}$ is Cauchy in the metric space $\left(X, d_{1}\right)$ then $\left\{x_{n}\right\}$ is Cauchy in the metric space $(X, d)$
    \end{enumerate}
\end{lemma}

\begin{corollary}
    Let $X$ be a vector space and let $\|\cdot\|$ and $\|\cdot\|_{1}$ be equivalent norms on $X .$ Let $d$ and $d_{1}$ be the metrics defined by $d(x, y)=\|x-y\|$ and $d_{1}(x, y)=\|x-y\|_{1}$ Let $\left\{x_{n}\right\}$ be a sequence in $X$.
    \begin{enumerate}
        \item $\left\{x_{n}\right\}$ converges to $x$ in the metric space $(X, d)$ $\iff$ $\left\{x_{n}\right\}$ converges to $x$ in the metric space $\left(X, d_{1}\right)$.
        \item $\left\{x_{n}\right\}$ is Cauchy in the metric space $(X, d)$ $\iff$ $\left\{x_{n}\right\}$ is Cauchy in the metric space $\left(X, d_{1}\right)$.
        \item $(X, d)$ is complete $\iff$ $\left(X, d_{1}\right)$ is complete.
    \end{enumerate}
\end{corollary}

\begin{theorem}
    Let $X$ be a finite-dimensional vector space with norm $\|\cdot\|$ and let $\left\{e_{1}, e_{2}, \ldots, e_{n}\right\}$ be a basis for $X .$ Another norm on $X$ was defined by
    $$
        \left\|\sum_{j=1}^{n} \lambda_{j} e_{j}\right\|_{1}=\left(\sum_{j=1}^{n}\left|\lambda_{j}\right|^{2}\right)^{\frac{1}{2}}
    $$
    The norms $\|\cdot\|$ and $\|\cdot\|_{1}$ are equivalent.
\end{theorem}

\begin{corollary}
    If $\|\cdot\|$ and $\|\cdot\|_{2}$ are any two norms on a finite-dimensional vector space $X$ then they are equivalent.
\end{corollary}

\begin{lemma}
    Let $X$ be a finite-dimensional vector space over $\mathbb{F}$ and let $\left\{e_{1}, e_{2}, \ldots, e_{n}\right\}$ be a basis for $X .$ If $\|\cdot\|_{1}: X \rightarrow \mathbb{R}$ is the norm on $X$ defined by $\left\|\sum_{j=1}^{n} \lambda_{j} e_{j}\right\|_{1}=\left(\sum_{j=1}^{n}\left|\lambda_{j}\right|^{2}\right)^{\frac{1}{2}}$ then $X$ is a complete metric space.
\end{lemma}

\begin{corollary}
    If $\|\cdot\|$ is any norm on a finite-dimensional space $X$ then $X$ is a complete metric
    space.
\end{corollary}

\begin{corollary}
    If $Y$ is a finite-dimensional subspace of a normed vector space $X,$ then $Y$ is closed.
\end{corollary}

\subsection{Banach Spaces}

\par In infinite-dimensional vector space $X$, many methods we discussed in the last section will expire.

\begin{lemma}
    If $X$ is a normed vector space and $S$ is a linear subspace of $X$ then $\bar{S}$ is a linear subspace of $X$.
\end{lemma}


\begin{definition}
    Let $X$ be a normed vector space and let $E$ be any non-empty subset of $X .$ The closed linear span of $E,$ denoted by $\overline{Sp} E,$ is the intersection of all the closed linear subspaces of $X$ which contain $E$.
\end{definition}

\begin{lemma}
    Let $X$ be a normed space and let $E$ be any non-empty subset of $X$.
    \begin{enumerate}
        \item $\overline{\operatorname{Sp}} E$ is a closed linear subspace of $X$ which contains $E$.
        \item $\overline{\mathrm{Sp}} E=\overline{\operatorname{Sp} E},$ that is, $\overline{\operatorname{Sp}} E$ is the closure of $\operatorname{Sp} E$
    \end{enumerate}
\end{lemma}

\begin{theorem}[Riesz’ Lemma]
    Suppose that $X$ is a normed vector space, $Y$ is a closed linear subspace of $X$ such that $Y \neq X$ and $\alpha$ is a real number such that $0<\alpha<1 .$ Then there exists $x_{\alpha} \in X$ such that $\left\|x_{\alpha}\right\|=1$ and $\left\|x_{\alpha}-y\right\|>\alpha$ for all $y \in Y$
\end{theorem}

\begin{theorem}
    If $X$ is an infinite-dimensional normed vector space then neither $D=\{x \in X:$ $\|x\| \leq 1\}$ nor $K=\{x \in X:\|x\|=1\}$ is compact.
\end{theorem}

\begin{definition}[Banach Space]
    A Banach space is a normed vector space which is complete under the metric
    associated with the norm.
\end{definition}

\begin{theorem}
    \begin{itemize}
        \item Any finite-dimensional normed vector space is a Banach space.
        \item If $X$ is a compact metric space then $C_{\mathrm{F}}(X)$ is a Banach space.
        \item If $(X, \Sigma, \mu)$ is a measure space then $L^{p}(X)$ is a Banach space for $1 \leq p \leq \infty$.
        \item $\ell^{p}$ is a Banach space for $1 \leq p \leq \infty$.
        \item If $X$ is a Banach space and $Y$ is a linear subspace of $X$ then $Y$ is a Banach space if and only if $Y$ is closed in $X$.
        \item
    \end{itemize}
\end{theorem}

\begin{theorem}
    Let $X$ be a Banach space and let $\left\{x_{n}\right\}$ be a sequence in $X$. If the series $\sum_{k=1}^{\infty}\left\|x_{k}\right\|$ converges ($\sum_{k=1}^{\infty} x_{k}=\lim _{n \rightarrow \infty} s_{n}$ exists) then the series $\sum_{k=1}^{\infty} x_{k}$ converges.
\end{theorem}

\section{Inner Product Spaces, Hilbert Spaces}

\subsection{Inner Products}
\begin{definition}
    Let $X$ be a vector space. An \textbf{inner product} on $X$ is a function $(\cdot, \cdot):$ $X \times X \rightarrow \mathbb{R}$ such that for all $x, y, z \in X$ and $\alpha, \beta \in \mathbb{R}$
    \begin{itemize}
        \item $(x, x) \geq 0$ ($\in R$ if complex)
        \item $(x, x)=0$ if and only if $x=0$
        \item $(\alpha x+\beta y, z)=\alpha(x, z)+\beta(y, z)$
        \item $(x, y)=(y, x)$. ($(x, y)=\overline{(y, x)}$ for complex vector space)
    \end{itemize}
\end{definition}

\begin{lemma}
    Let $X$ be an inner product space, $x, y \in X $:
    \begin{itemize}
        \item $|(x, y)|^{2} \leq(x, x)(y, y), x, y \in X$
        \item the function $\|\cdot\|: X \rightarrow \mathbb{R}$ defined by $\|x\|=(x, x)^{1 / 2},$ is a norm on $X$.
    \end{itemize}
\end{lemma}

\begin{theorem}
    To prove a norm:
    \begin{itemize}
        \item $ \|x+y\|^{2}+\|x-y\|^{2}=2\left(\|x\|^{2}+\|y\|^{2}\right) $ (the parallelogram rule)
        \item $4(x, y)=\|x+y\|^{2}-\|x-y\|^{2}+i\|x+i y\|^{2}-i\|x-i y\|^{2}$ (the polarization identity)
    \end{itemize}
\end{theorem}

\begin{lemma}
    If $\lim _{n \rightarrow \infty} x_{n}=x, \lim _{n \rightarrow \infty} y_{n}=y$, then $\lim _{n \rightarrow \infty}\left(x_{n}, y_{n}\right)=(x, y)$
\end{lemma}

\subsection{Orthogonality}

\begin{definition}
    Orthogonal: $(x, y)=0$, Orthonormal basis:$\left\{e_{1}, \ldots, e_{k}\right\}$ that $x=\sum_{n=1}^{k}\left(x, e_{n}\right) e_{n}$
\end{definition}

\begin{lemma}
    $Gram-Schmidt$ algorithm: $b_{k+1}=v_{k+1}-\sum_{n=1}^{k}\left(v_{k+1}, e_{n}\right) e_{n}, \quad e_{k+1}=\frac{b_{k+1}}{\left\|b_{k+1}\right\|}$
\end{lemma}

\begin{theorem}
    Orthonormal basis then: $\left\|\sum_{n=1}^{k} \alpha_{n} e_{n}\right\|^{2}=\sum_{n=1}^{k}\left|\alpha_{n}\right|^{2}$
\end{theorem}

\begin{definition}
    An inner product space which is complete with respect to the metric associated with the norm induced by the inner product is called a \textbf{Hilbert space}. (eg. every finite-dimensional inner product space, $L2$ and $\ell^{2}$ with standard inner product)
\end{definition}

\begin{lemma}
    $Y \subset \text{Hilbert space } \mathcal{H}$ is a linear subspace, then $Y$ is a Hilbert space if and only if $Y$ is closed.
\end{lemma}

\subsection{Orthogonal Complements}

\begin{definition}
    $X$ be an inner product space and let $A$ be a subset of $X .$ The orthogonal complement of $A$ is the set
    $$
        A^{\perp}=\{x \in X:(x, a)=0 \text { for all } a \in A\}
    $$
\end{definition}

\begin{lemma}
    If $X$ is an inner product space and $A \subset X$ then:
    \begin{itemize}
        \item $0 \in A^{\perp}$.
        \item If $0 \in A$ then $A \cap A^{\perp}=\{0\},$ otherwise $A \cap A^{\perp}=\emptyset$
        \item $\{0\}^{\perp}=X ; X^{\perp}=\{0\}$
        \item If $A$ contains an open ball $\bar{B}_{a}(r),$ for some $a \in \bar{X}$ and some positive $r>\overline{0}$ then $\Lambda^{\perp}=\{0\} ;$ in particular, if $A$ is a non-cmpty open sct then $\Lambda^{\perp}=\{0\}$
        \item If $B \subset A$ then $A^{\perp} \subset B^{\perp}$.
        \item $A^{\perp}$ is a closed linear subspace of $X$.
        \item $A \subset\left(A^{\perp}\right)^{\perp}$.
    \end{itemize}
\end{lemma}

\begin{lemma}
    Let $Y$ be a linear subspace of an inner product space $X .$ Then
    $$
        x \in Y^{\perp} \Longleftrightarrow\|x-y\| \geq\|x\|, \quad \forall y \in Y
    $$
\end{lemma}

\begin{definition}
    A subset $A$ of a vector space $X$ is \textbf{convex} if, for all $x, y \in A$ and all $\lambda \in[0,1]$ $\lambda x+(1-\lambda) y \in A$
\end{definition}

\begin{theorem}
    $A$ non-empty, closed, convex subset of a Hilbert space $\mathcal{H}$ and $p \in \mathcal{H}$. There exists a unique $q \in A$ such that
    $$
        \|p-q\|=\inf \{\|p-a\|: a \in A\}
    $$
\end{theorem}

\begin{theorem}(Orthogonal decomposition)
    $Y$ a closed linear subspace of a Hilbert space $\mathcal{H}$. For any $x \in \mathcal{H},$ there exists a unique $y \in Y$ and $z \in Y^{\perp}$ such that $x=y+z .$ Also $,\|x\|^{2}=\|y\|^{2}+\|z\|^{2}$. 
\end{theorem}

\begin{lemma}
    If $Y$ is a closed linear subspace of a Hilbert space $\mathcal{H}$ then $Y^{\perp \perp}=Y$, If $Y$ is any linear subspace of a Hilbert space $\mathcal{H}$ then $Y^{\perp \perp}=\bar{Y}$.
\end{lemma}



\end{document}