\documentclass[10pt]{yerbaformat}
\title{随机过程引论 Notes\footnote{EEEEEErin~}}
\date{}

\begin{document}
\author{Yuling Shi}
\maketitle
% \tableofcontents
\footnotesize

% \section{基本概念}
% \begin{definition}[矩母函数]
%     对千随机变量 $X \sim F$. 若下面的数学期望存在, 则称$$M_{X}(t):=\mathbb{E}\left[\mathrm{e}^{t X}\right]=\int_{\mathbb{R}} \mathrm{e}^{t x} \mathrm{d} F(x), \quad t \in \mathbb{R}$$为 $X$ 的矩母函数.矩母函数唯一确定$X$的分布,可以通过$M_{X}(t)$ 求出$X$的各阶矩$$\mathbb{E}\left[X^{n}\right]=M_{X}^{(n)}(0), \quad n \geq
%         1,$$ 正态分布的矩母函数为$M_{X}t=e^{\mu t+\frac{t^{2} \sigma^{2}}{2}}$
% \end{definition}
% \par 对于相互独立的随机变量有$M_{X+Y}(t)=M_{X}(t) \cdot M_{Y}(t)$.

% \begin{definition}[数列的母函数(痛苦回忆)]
%     设有一个实数列 $a=\left\{a_{0}, a_{1},\cdots, a_{n}, \cdots\right\},$ 如果幂级数
%     $$
%         G_{a}(z):=a_{0}+a_{1} z+a_{2} z^{2}+\cdots+a_{n} z^{n}+\cdots
%     $$
%     在$0$点的一个非空领域上收敛, 则称它为数列 a 的母函数.
% \end{definition}

% \begin{definition}[随机变量的母函数]
%     设 $\xi$ 是 $\mathbb{Z}_{+-}$ 值随机变量,,则其概率分布律是一个有界数列, 定义$\xi$的母函数:$$
%         G_{\xi}(z):=\sum z^{n} \mathbb{P}(\xi=n)=\mathbb{E}\left[z^{\xi}\right]
%     $$
% \end{definition}
% \par 随机变量的母函数同样有$G_{\xi+\eta}(z)=G_{\xi}(z) G_{\eta}(z)$

% \begin{definition}[随机过程]
%     $\forall t \in T, \quad X(t) \equiv X(t, \omega): \Omega \mapsto \mathbb{R}$为随机变量,程随机变量族$X=\{X(t), t \in T\}$为随机过程.$X$可能取到的值称为状态,全体状态$E$称为状态空间.
% \end{definition}

\section{泊松过程}

\begin{theorem}
    设 $N \sim P(\lambda), N$ 个事件独立地为第 $i$ 个类型, 其 $p_{1}+p_{2}+\cdots p_{n}=1 .$ 记 $N_{i}$ 为 第 $i$ 类事件发生的次数, 则 $N_{i} \sim P\left(\lambda p_{i}\right),$ 且 $N_{1}, N_{2}, \cdots, N_{n}$ 相互独立.
\end{theorem}

\begin{definition}[点过程]
    $(\Omega, \mathcal{F}, \mathbb{P})$ 上随机过程 $\{N(t), t \geq 0\}$ 如果满足:
    $\forall \omega \in \Omega$, 其样本函数 $t \mapsto N(t) \equiv N(t, \omega)$
    以概率 1 是一个只取非负整数值的, 右连续的, 单调增函数, 则称 $\{N(t), t \geq 0\}$ 为点过程.
\end{definition}

\begin{definition}[Poisson 点过程]
    如果计数过程 $\{N(t), t \geq 0\}$ 满足:
    \begin{enumerate}
        \item $N(0)=0$ a.s.
        \item $N(t)$ 是独立增量过程: 对任意 $t, s \geq 0, N(t+s)-N(s)$ 与 $\mathcal{F}_{s}$ 独立
        \item $N(t)$ 的其增量服从 Poisson 分布: 对任意 $t, s \geq 0$, $\mathbb{P}(N(t+s)-N(s)=k)=\frac{(\lambda t)^{k}}{k !} \mathrm{e}^{-\lambda t}, k=0,1,2, \cdots$
    \end{enumerate}
    则称 $\{N(t), t \geq 0\}$ 为具参数 $\lambda$ 的 Poisson 过程.
\end{definition}

\begin{definition}
    设 $\{N(t), t \geq 0\}$ 为参数为 $\lambda$ 的 Poisson 过程. 记 $X_{1}$ 为第一个事 件的来到时刻, $X_{n}$ 为第 $n-1$ 个到第 $n$ 个事件之间的时 间 $(n \geq 2), S_{n}$ 为第 $n$ 个事件来到的时间 $($ 或等待时间 $)(n \geq 1)$: $S_{n}:=\inf \left\{t: t>S_{n-1}, N(t)=n\right\}, \quad S_{0} \equiv 0$ 则 $X_{n}=S_{n}-S_{n-1}(n \geq 1), \quad S_{n}=\sum_{i=1}^{n} X_{i}(n \geq 1)$ 称 $\left\{X_{n}, n \geq 1\right\}$ 为来到时间间隔序列, $\left\{S_{n}, n \geq 1\right\}$ 为等待时间序列.
\end{definition}

\begin{lemma}
    $\{N(t) \geq n\}=\left\{S_{n} \leq t\right\}$ \\
    $\{N(t)=n\}=\left\{S_{n} \leq t<S_{n+1}\right\}=\left\{S_{n} \leq t\right\} \backslash\left\{S_{n+1} \leq t\right\}$
\end{lemma}

\begin{lemma}[来到时间间隔分布]
    $X_{n}(n=1,2, \cdots)$ 为 i.i.d., 均值为 $1 / \lambda$ 的指数分布随机变量序列
\end{lemma}

\begin{lemma}[等待时间分布]
    $S_{n} \sim \Gamma(n, \lambda),$ 即其密度函数为 $\lambda \mathrm{e}^{-\lambda t} \frac{(\lambda t)^{n-1}}{(n-1) !}$ 特别的, $X_{1} \equiv S_{1} \sim \operatorname{Exp}(\lambda)$.
\end{lemma}

\begin{lemma}
    在 $[0, t)$ 中有一个事件发生的条件下, 这个事件发生的时间 $X_{1}$ 是个均匀分布的随机变量.
\end{lemma}

\begin{theorem}[Poisson 过程完全随机性]
    在已知 $N(t)=n$ 的条件下, $n$ 个顾客的来到时刻 $S_{1}, \cdots, S_{n}$ 与 $n$ 个 $[0, t]$ 上均匀分布的独立随机变量的顺序统计量有相同分布. 即 $f_{S_{1}, \cdots, S_{n} \mid N(t)=n}\left(s_{1}, \cdots, s_{n} \mid n\right)=\frac{n !}{t^{n}}, 0<s_{1}<\cdots<s_{n} \leq $
\end{theorem}

\begin{theorem}[Poisson 过程分流]
    设顾客按强度 $\lambda$ 的 Poisson 过程到达. $N_{i}(t)$ 为 $[0, t]$ 中到达的第新 $\mathrm{i}$ 类顾客数 $(i=1,2) .$ 假定时刻 $s$ 的到达者与其他的到达独立. 令
    $p(s):=\mathbb{P}(s$ 时刻到达者为第 1 类 $)$, $1-p(s):=\mathbb{P}(s$ 时刻到达者为第 2 类 $),$ 则 $N_{1}(t), N_{2}(t)$ 为独立的随机变量且 分别服从参数为 $\lambda t p, \lambda t(1-p)$ 的 Poisson 分布. 其中 $p=\frac{1}{t} \int_{0}^{t} p(s) \mathrm{d} s$
\end{theorem}

\begin{lemma}
    对于任意 $[0, \infty)$ 上的可积函数 $f$, 有 $\mathbb{E}\left[\sum_{n=1}^{\infty} f\left(S_{n}\right)\right]=\lambda \int_{0}^{\infty} f(t) \mathrm{d} t$
\end{lemma}

\begin{definition}[复合 Poisson 过程]
    若对任意 $t \geq 0,$ 有 $X(t)=\sum_{i=1}^{N(t)} Y_{i .}$ 其中, $Y_{1}, Y_{2}, \cdots$ 为 i.i.d. 且与参数为 $\lambda$ 的 Poisson 过程 $\{N(t), t \geq 0\}$ 独立, 则称随机过 程 $X=\{X(t), t \geq 0\}$ 是复合 Poisson 过程
\end{definition}

\begin{theorem}
    复合 Poisson 过程 $X(t)=\sum_{i=1}^{N(t)} Y_{i}$ 有平稳独立增量, 矩母函数 $\phi_{t}(u) \equiv \mathbb{E}\left[\mathrm{e}^{u X(t)}\right]=\exp \left\{\lambda t\left(\phi_{Y}(u)-1\right)\right\}$ , $$\begin{aligned} \mathbb{E}[X(t)] &=\phi_{t}^{\prime}(0)=\lambda t \cdot \mathbb{E} Y \\ \mathbf{D}(X(t)) &=\phi_{t}^{\prime \prime}(0)-\left(\phi_{t}^{\prime}(0)\right)^{2}=\lambda t \cdot \mathbb{E} Y^{2} \end{aligned}$$
\end{theorem}


\section{更新过程}
\begin{definition}
    设 $\left\{X_{n}, n \geq 1\right\}$ 为独立同分布的非负随机变量序列
    $$
        S_{0} \equiv 0, S_{n}:=\sum_{i=1}^{n} X_{i}, \quad(n \geq 1)
    $$
    定义更新过程 $N(t)=\sup \left\{n: S_{n} \leq t\right\}=\sum_{n=1}^{\infty} I_{\left\{S_{n} \leq t\right\}}:(0, t]$ 的更新次数
\end{definition}

\par $N$ 的分布函数为 $\mathbb{P}(N(t)=n)=\mathbb{P}\left(S_{n} \leq t\right)-\mathbb{P}\left(S_{n+1} \leq t\right)=F_{n}(t)-F_{n+1}(t)$, 其期望 $m(t)=\mathbb{E}\left[\sum_{n \geq 1} 1_{\left\{S_{n} \leq t\right\}}\right]=\sum_{n \geq 1} \mathbb{P}\left(S_{n} \leq t\right)=\sum_{n=1}^{\infty} F_{n}(t)\leq \frac{F(t)}{1-F(t)}<\infty$

\begin{theorem}[趋近无穷速度]
    $\mathbb{P}\left(\lim _{t \rightarrow \infty} \frac{1}{t} N(t)=\frac{1}{\mu}\right)=1$
\end{theorem}

\begin{definition}[离散型停时]
    设 $N$ 为非负整数值随机变量, $X_{1}, X_{2}, \cdots$ 为任意随机变量序列 若对任意 $n=$ 有 $\{N=n\}$ 关于 $\sigma\left(X_{1}, X_{2}, \cdots, X_{n}\right)$ 可测则称 $N$ 为关于 $\left\{X_{n}\right\}$ 的停时 / Markov 时间.
\end{definition}

\begin{theorem}[Wald 等式]
    设 $X_{1}, \cdots, X_{n}$ 为 i.i.d. 随机变量序列且 $\mathbb{E} X_{1}<\infty, N$ 为关于 $X_{1}, \cdots, X_{n}$ 的停时且 $\mathbb{E} N<\infty,$ 则 $\mathbb{E} \sum_{n=1}^{N} X_{n}=\mathbb{E} N \cdot \mathbb{E} X_{1}$
\end{theorem}

\begin{corollary}
    若 $\mu=\mathbb{E} X_{1}<\infty,$ 则 $\mathbb{E}\left[\sum_{n=1}^{N(t)+1} X_{n}\right]=\mathbb{E} X \cdot \mathbb{E}[N(t)+1]=\mu(m(t)+1)$
\end{corollary}

\begin{theorem}[基本更新定理]
    $\lim _{t \rightarrow \infty} \frac{1}{t} m(t)=\frac{1}{\mu} \quad\left(\right.$ 设 $\left.\frac{1}{\infty}=0\right)$
\end{theorem}

\begin{theorem}[中心极限定理]
    假设 $\mu, \sigma^{2}$ 均有限, 则当 $t \rightarrow \infty$ 时, $N(t)$ 的近似分布是正态分布 $ N\left(t / \mu, t \sigma^{2} / \mu^{3}\right) $
\end{theorem}

\section{Markov 链}

\begin{definition}[无后效性]
    已知现在, 将来与过去无关. $\mathbb{P}\left(X_{n+1}=j \mid X_{n}, \cdots, X_{1}\right)=\mathbb{P}\left(X_{n+1}=j \mid X_{n}\right)$ 称为离散时间 Markov 链, 进一步如果 $P_{i j}(n):=\mathbb{P}\left(X_{n+1}=j \mid X_{n}=i\right)$ 与 $n$ 无关称为\textbf{时齐 Markov 链}
\end{definition}

% \par 转移矩阵 $P_ij$ 满足非负性和正则性 $\forall i \in E, \sum_{j \in E} P_{i j}=1$ , 今后只讨论时齐 Markov 链故以下略去"时齐". 对于当前研究的 Markov 链利用时齐性质可以将有限维分布刻画为单步转移概率的乘积 $$\mathbb{P}\left(X_{n_{1}}=x_{1}, \cdots, X_{n_{k}}=x_{k}\right)=\mathbb{P}\left(X_{n_{1}}=x_{1}\right) \mathbb{P}\left(X_{n_{2}}=x_{2} \mid X_{n_{1}}=x_{1}\right) \cdots \mathbb{P}\left(X_{n_{k}}=x_{k} \mid X_{n_{k-1}}=x_{k-1}\right)$$.

\begin{theorem}
    若以 $S$ 为状态空间的随机过程 $\left\{X_{n}, n \geq 0\right\}$ 满足如下两个条件:
    \begin{enumerate}
        \item $\left\{\xi_{n}, n \geq 1\right\}$ 是独立同分布的随机序列且 $X_{0}$ 与 $\left\{\xi_{n}, n \geq 1\right\}$ 也相互独立
        \item 存在函数 $f: S \times S \rightarrow S$ 使得 $X_{n}=f\left(X_{n-1}, \xi_{n}\right), n \geq 1 .$
    \end{enumerate}
    那么 $\left\{X_{n}, n \geq 0\right\}$ 是马氏链, 其转移概率为 $p_{i j}=\mathbb{P}\left(f\left(i, \xi_{1}\right)=j\right), i, j \in S$
\end{theorem}

\begin{theorem}[Chapman-Kolmogorov 方程]
    $\mathbf{P}^{(n+m)}=\mathbf{P}^{(n)} \cdot \mathbf{P}^{(m)}$
\end{theorem}

\begin{definition}[可达]
    若存在 $n \geq 0$ 使得 $p_{i j}^{(n)}>0,$ 则称 $i$ 到 $j$ \textbf{可达}, 记作 $i \rightarrow j .$ 若 $i \rightarrow j$ 且 $j \rightarrow i,$ 则为相通, 记作 $i \leftrightarrow j$. 相应衍生出自反性、对称性和传递性等性质.
\end{definition}

\begin{definition}[可约]
    若某类中的任何状态均不可达其它类中任一状态, 则称此类是\textbf{闭}的. 若 Markov 链的状态空间只存在一个类, 即一切状态彼此相通, 则称该 Markov 链\textbf{不可约}.
\end{definition}

\begin{definition}[周期]
    若集合 $\left\{n \geq 1: p_{i i}^{(n)}>0\right\}$ 非空, 则称该集合的最大公约数 $d(i):=G C D\left\{n \geq 1: p_{i i}^{(n)}>0\right\}$ 为 $i$ 的周期. 
    \begin{enumerate}
        \item $d(i)>1$ 称 $i$ 是周期的
        \item $d(i)=1$ 称 $i$ 是非周期的
    \end{enumerate}
\end{definition}

\begin{lemma}
    若 $i \leftrightarrow j$ 则常返性与周期性都相同. 若 $i$ 常返且 $i \leftrightarrow j,$ 则 $f_{j i}=1 .$
\end{lemma}

\begin{definition}[首达时间]
    定义第一次完成两个状态之间转移的时间 $T_{i j}=\min \left\{n: X_{0}=i, X_{n}=j, n \geq 0\right\}$ 称为\textbf{首达时间} % , 若达不到则取 $T_{ij}=+\infty$
\end{definition}

\begin{definition}
    对任意 $n,$ 记 $f_{i j}^{n}$ 为 $i$ 发经 $n$ 步首次到达 $j$ 的概率: $f_{i j}^{n}:=\mathbb{P}\left(X_{n}=j, X_{k} \neq j, k=1,2, \cdots, n-1 \mid X_{0}=i\right), f_{i j}^{0} \equiv 0 $, 从而有 $i$ 发经有限步\textbf{最终到达} $j$ 的概率 $f_{i j}:=\sum_{n=1}^{\infty} f_{i j}^{n}=\mathbb{P}\left(T_{i j}<\infty\right)$
    \begin{enumerate}
        \item 若 $f_{i i}=1$ 则称 $i$ 为\textbf{常返状态}
        \item 若 $f_{i i}<1$ 则称 $i$ 为\textbf{暂留状态}
    \end{enumerate}
\end{definition}

\begin{definition}[判定定理]
    常返的充要条件为 $\sum_{n=0}^{\infty} p_{i i}^{(n)}=\infty$, 暂留的充要条件是 $\sum_{n=0}^{\infty} p_{i i}^{(n)}=1 /\left(1-f_{i i}\right)<\infty$
\end{definition}

\begin{theorem}[Polya 定理]
    $d \geq 3$ 维空间中的对称随机游动是非常返的, $d=1,2$ 维空间中的对称随机游动都常返, 且都是零常返的.
\end{theorem}

\begin{definition}[正常返与零常返]
    记平均回转时间 $\mu_{i i}:=\sum_{i=1}^{\infty} n f_{i i}^{n}$
    \begin{enumerate}
        \item 若 $\mu_{i i}<\infty$ 则称 $i$ 为\textbf{正常返}
        \item 若 $\mu_{i i}=\infty$ 则称 $i$ 为\textbf{零常返} (无穷可列多个状态时才可能出现)
    \end{enumerate}
    非周期正常返的状态 $j$ 称为\textbf{遍历状态}.
\end{definition}

% \begin{theorem}
%     对任意 $i, j \in E, n \geq 1$:
%     \begin{enumerate}
%         \item $p_{i j}^{(n)}=\sum_{l=1}^{n} f_{i j}^{1} P_{j j}^{n-l}$
%         \item $f_{i j}^{n}=\sum_{k \neq j} P_{i k} f_{k j}^{n-1} \cdot I_{\{n>1\}}+P_{i j} \cdot I_{\{n=1\}},$ 即 $\left\{\begin{array}{l}f_{i j}^{1}=P_{i j} \\ f_{i j}^{n}=\sum_{k \neq j} P_{i k} f_{k j}^{n-1} \quad(n=2,3, \cdots)\end{array}\right.$
%         % \item $i \rightarrow j \Leftrightarrow f_{i j}>0$,
%         %       $i \leftrightarrow j \Leftrightarrow f_{i j}>0$ 且 $f_{j i}>0 .$
%     \end{enumerate}
% \end{theorem}

% \begin{corollary}
%     若 $j$ 非常返, 则 $\sum_{n=1}^{\infty} p_{i j}^{(n)}<\infty, \lim _{n \rightarrow \infty} p_{i j}^{(n)}=0 $; 若 $j$ 常返则 $\lim _{n \rightarrow \infty} p_{i j}^{(n)}=0 $ 与零常返等价. 
% \end{corollary}

\begin{theorem}[基本极限定理]
    \ 
    \begin{enumerate}
        \item 若 $i$ 为非常返或零常返态, 则 $\lim _{n \rightarrow \infty} p_{i i}^{(n)}=0 .$
        \item 若 $i$ 为周期为 $d$ 的常返态, 则有 $\lim _{n \rightarrow \infty} P_{i i}^{(n d)}=\frac{d}{\mu_{i i}}$.
              其中当 $\mu_{i i}=\infty$ 时, 右式设为 $0.$
        \item 若 $i$ 为非周期正常返态, 则有 $\lim _{n \rightarrow \infty} p_{i i}^{(n)}=\frac{1}{\mu_{i i}}$.
    \end{enumerate}
\end{theorem}

\begin{definition}[平稳分布]
    若非负数列 $\left\{\pi_{j}\right\}$ 满足 $\sum_{j \in E} \pi_{j}=1$ 且 $\pi = \pi P$
    % \begin{enumerate}
    %     \item 
    %     % \item $\pi_{j}=\sum_{i \in E} \pi_{i} P_{i j}, \forall j \in E$.
    %     \item 
    % \end{enumerate}
    则称 $\left\{\pi_{j}\right\}$ 为 $X$ 的\textbf{平稳分布}. 初始分布是平稳分布则称为\textbf{平稳过程}.
\end{definition}

\begin{lemma}
    如果 Markov 链是不可约的, $j$ 是非周期的, 则 $\lim _{n \rightarrow \infty} p_{i j}^{(n)}=\frac{1}{\mu_{j j}} = \pi _{j}>0$ 称 $\left\{\frac{1}{\mu_{i j}}\right\}$ 为\textbf{极限分布}.
\end{lemma}

\begin{theorem}
    非周期不可约链是正常返的充要条件是该链存在平稳分布, 此平稳分布就是它的极限分布. 有限状态的马尔可夫链总存在平稳分布.
\end{theorem}

\par 综上所述有:
\begin{enumerate}
    \item $i$ 是零常返 ($\mu_{i i}=\infty$) $\Leftrightarrow \sum_{n} p_{i i}^{(n)}=\infty$ 且 $\lim _{n \rightarrow \infty} p_{i i}^{(n)}=0$
    \item $i$ 是正常返 ($\mu_{i i}<\infty$) $\Leftrightarrow \sum_{n} p_{i i}^{(n)}=\infty$ 且 $\varlimsup_{n \rightarrow \infty} p_{i i}^{(n)}>0$
    % \item $i$ 是遍历 (非周期正常返) $\Leftrightarrow \sum_{n} p_{i i}^{(n)}=\infty$ 且 $\lim _{n \rightarrow \infty} p_{i i}^{(n)}=\frac{1}{\mu_{i i}}>0$
    \item $i$ 是遍历 (非周期正常返) $\Leftrightarrow \lim _{n \rightarrow \infty} p_{i i}^{(n)}=\frac{1}{\mu_{i i}}>0$
    \item $i$ 是非常返 $\Leftrightarrow \sum_{n} p_{i i}^{(n)}<\infty $
\end{enumerate}

\section{离散鞅论}
% \subsection{基本概念}

\begin{definition}[鞅]
    定义$\mathcal{F}_{n}^{0}:=\sigma\left(Z_{k}, k \leq n\right)$, 若 $\mathbb{E}\left|Z_{n}\right|<\infty$, 且满足 $\mathbb{E}\left[Z_{n+1} \mid \mathcal{F}_{n}^{0}\right]=Z_{n}$ 则称 $Z$ 为鞅. $\mathbb{E}\left[Z_{n+1} \mid \mathcal{F}_{n}\right] \geq(\leq) Z_{n}, n \geq 0$ 则 $Z$ 称为下 $($ 上)鞅.
\end{definition}

\par 迭代期望公式: 若 $\mathcal{A} \in \mathcal{B}$ 有 $\mathbb{E}(\mathbb{E}(\xi \mid \mathcal{A}) \mid \mathcal{B})=\mathbb{E}(\mathbb{E}(\xi \mid \mathcal{B}) \mid \mathcal{A})=\mathbb{E}(\xi \mid \mathcal{A})$ , 双期望公式: $\mathbb{E}[\mathbb{E}[X \mid Y]]=\mathbb{E}\left[{X}\right]$

% \subsection{鞅基本定理}

\begin{definition}[停时]
    如果值域为 $I$ 的随机变量 $T$ 满足 $\{T=n\} \in \mathcal{F}_{n}, n \in I$, 则称为停时.
\end{definition}

\begin{lemma}
    如果 $T$ 与 $S$ 是停时, 那么 $T \wedge S$ 与 $T \vee S$ 也是停时.
\end{lemma}

\begin{definition}[停止过程]
    定义 $T-$ 停止序列 $X_{n}^{T}(\omega):=X_{n \wedge T}(\omega), n \geq 0$ 称之为停止过程.
\end{definition}

\begin{theorem}[Doob 有界停止定理]
    \
    \begin{enumerate}
        \item 如果 $\left\{X_{n}: n \in I\right\}$ 是鞅, $T$ 是停时, 那么鞅的停止序列 $\left\{X_{n}^{T}: n \in I\right\}$ 也是鞅. 进一步, 如果 $T$ 是有界的, 那么 $\mathbb{E} X_{T}=\mathbb{E} X_{0}$.
        \item 设 $X$ 是下鞅, $S, T$ 是停时且 $S \leq T,$ 则 $\left\{X_{n}^{T}-X_{n}^{S}: n \in I\right\}$ 是下鞅. 因此 $\mathbb{E} X_{n}^{S} \leq \mathbb{E} X_{n}^{T}$
    \end{enumerate}
\end{theorem}

\begin{theorem}[Wald 等式]
    设 $\left\{\xi_{n}: n \geq 1\right\}$ 是可积独立同分布随机序列且 $\mathbb{E}_{\zeta 1}^{\tau}=0$, $T$ 是可积停时, 则 $\mathbb{E} \sum_{n=1}^{T} \xi_{n}=0 .$
\end{theorem}

\begin{theorem}[停止定理]
    设 $\left(M_{n}, \mathcal{F}_{n}\right)$ 是鞅, $T$ 是停时且满足
    \begin{enumerate}
        \item $\mathbb{P}(T<\infty)=1$
        \item $\mathbb{E}\left[\left|M_{T}\right|\right]<\infty$
        \item $\lim _{n \rightarrow \infty} \mathbb{E}\left[\left|M_{n}\right| ; T>n\right]=0$
    \end{enumerate}
    则有 $\mathbb{E} M_{T}=\mathbb{E} M_{0}$.
\end{theorem}

\section{Brown 运动}

\begin{definition}[Brown 运动]
    若 $\{X(t), t \geq 0\}$ 满足
    \begin{enumerate}
        \item $X(0)=0,$ a.s.
        \item $X(t)$ 是独立增量过程
        \item 对任意 $s, t>0, X(s+t)-X(s) \sim N\left(0, c^{2} t\right)$
        \item $X(t)$ 是关于 $t$ 的连续函数. 则称 $\{X(t), t \geq 0\}$ 为 Brown 运动
    \end{enumerate}
    当 $c=1$ 时, 称 $\{X(t), t \geq 0\}$ 为\textbf{标准 Brown 运动}, Brown 运动是一个 Markov 过程.
\end{definition}

\begin{theorem}
    对任意 $0 \leq a<s, t<\infty$, 有$\mathbb{E}[(X(s)-X(a))(X(t)-X(a))]=(s-a) \wedge(t-a)$, 进一步 $\mathbb{E}[X(s) X(t)]=s \wedge t$.
\end{theorem}

\begin{theorem}[有限维分布]
    $\forall 0=t_{0}<t_{1}<\cdots<t_{n},\left(X\left(t_{1}\right), \cdots, X\left(t_{n}\right)\right)$ 的联合密度为 $$f_{t_{1}, \cdots, t_{n}}\left(x_{1}, \cdots, x_{n}\right)=\prod_{i=1}^{n} f_{t_{i}-t_{i-1}}\left(x_{i}-x_{i-1}\right)$$ 其中 $f_{t}(x)=\frac{1}{\sqrt{2 \pi t}} \mathrm{e}^{-\frac{x^{2}}{2 t}}, t>0, x \in \mathbf{R}$
\end{theorem}

\begin{definition}[Gauss 过程]
    若过程 $\{X(t), t \in T\}$ 对任意 $t_{1}<t_{2}<\cdots<t_{n}$, $
        \left(X\left(t_{1}\right), \cdots, X\left(t_{n}\right)\right)$ 的联合分布为 $n$ 维正态分布, 则称 $\{X(t), t \in T\}$ 为 \textbf{Gauss 过程}. Brown 运动是 Gauss 过程.
\end{definition}

\begin{theorem}
    设 $\{B(t), t \geq 0\}$ 是轨道连续的 Gauss 过程, $B(0)=0$ 且 $ \mathbb{E} B(t)=0, \mathbb{E}[B(s) B(t)]=t \wedge s(\forall s, t>0)$, 则 $\{B(t), t \geq 0\}$ 是 Brown 运动. 反之亦然.
\end{theorem}

\begin{theorem}[不变性]
    若 $\{B(t), t \geq 0\}$ 是 Brown 运动, $a, c>0,$ 则
    \begin{itemize}
        \item $\{B(t)-B(a) ; t \geq 0\}$ 是 Brown 运动
        \item $\{B(c t) / \sqrt{c} ; t \geq 0\}$ 是 Brown 运动
    \end{itemize}
\end{theorem}

% \begin{theorem}
%     $\tilde{X}(t):=\left\{\begin{array}{ll}t X(1 / t), & t>0 \\ 0, & t=0\end{array}\right.$ 是 Brown 运动.
% \end{theorem}

\begin{theorem}
    $\mathbb{P}\left(\sup _{t \geq 0} X(t)=\infty\right)=1$, $\mathbb{P}\left(\lim \sup _{t \geq 0} X(t)=\infty, \liminf _{t \geq 0} X(t)=-\infty\right)=1$
\end{theorem}

% \begin{lemma}
%     一维 Brown 运动是常返的, 以概率 $1$ 可达任何点.
% \end{lemma}













\end{document}